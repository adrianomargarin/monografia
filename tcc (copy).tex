\documentclass{iiufrgs}
\usepackage[latin1]{inputenc}
\usepackage{graphicx}
\usepackage[section]{placeins}
\usepackage{setspace}
\usepackage{fontenc}
\usepackage{listings}
\usepackage{color}
\usepackage{url}
\usepackage[printonlyused]{acronym}
\usepackage{rotating}
\usepackage{bytefield}
\usepackage[table]{xcolor}
\usepackage{multirow}
\usepackage{subfigure}
\usepackage{lscape}
\usepackage{enumitem}
\onehalfspacing

\hyphenation{cmdStart}
\hyphenation{cmdPublishService}
\hyphenation{cmdSearchIdentifiers}
\hyphenation{cmdSearchService}
\hyphenation{cmdConnect}
\hyphenation{cmdSend}
\hyphenation{cmdReceive}
\hyphenation{CALVERT}
\hyphenation{BETTSTETTER}
\hyphenation{ARCINIEGAS}
\hyphenation{DEITEL}
\hyphenation{MICROSOFT}

\setdescription{topsep=1em,parsep=0pt,partopsep=0pt,itemsep=0pt}
\setitemize{topsep=1em,parsep=0pt,partopsep=0pt,itemsep=0pt}
\setenumerate{topsep=1em,parsep=0pt,partopsep=0pt,itemsep=0pt}

\course{\cgcc}
\title{Portal de Algoritmos da Universidade de Caxias do Sul \\\large Evolu��o da ferramenta de gerenciamento do portal de algoritmos, visando substituir tecnologias defasadas}
\author{Margarin}{Adriano}
\advisor[]{Nascimento}{Alexandre Erasmo Krohn}
\coadvisor[]{Dornele}{Ricardo de Vargas}
\location{Caxias do Sul}{}
\bibpunct{(}{)}{;}{a}{,}{,}

\def\lstlistlistingname{Lista de Trechos de C�digo}
\def\lstlistingname{Trecho de C�digo}
\definecolor{lightgray}{rgb}{0.95,0.95,0.95}
\definecolor{darkgray}{rgb}{0.3,0.3,0.3}
\lstset{
	language=Java,
	escapeinside={(*@}{@*)},
	basicstyle=\scriptsize\color{black}\ttfamily,
	numbers=left,
	numberstyle=\scriptsize\ttfamily,
	stepnumber=1,
	numbersep=5pt,
	tabsize=4,
	extendedchars=true,
	breaklines=true,
	frame=single,
	keywordstyle=\color{black}\textbf,
	keywordstyle=[1]\color{black}\textbf,
	keywordstyle=[2]\color{black}\textbf,
	keywordstyle=[3]\color{black}\textbf,
	keywordstyle=[4]\color{black}\textbf,
	stringstyle=\color{black}\ttfamily,
	showspaces=false,
	showtabs=false,
	xleftmargin=17pt,
	framextopmargin=4pt,
	framexleftmargin=17pt,
	framexrightmargin=5pt,
	framexbottommargin=4pt,
	backgroundcolor=\color{lightgray},
	showstringspaces=false
}

\newcommand{\bitformattingDefault}[1]{%
	\tiny
	\ifnum#1=6$\cdots$\else#1\fi
}
\newcommand{\colorbitbox}[3]{%
	\rlap{\bitbox{#2}{\color{#1}\rule{\width}{\height}}}%
\bitbox{#2}{#3}}
\definecolor{colorPacote1}{rgb}{0.80,0.80,0.80}
\definecolor{colorPacote2}{rgb}{0.89,0.89,0.89}
\definecolor{colorPacote3}{rgb}{0.98,0.98,0.98}

\begin{document}

\maketitle

\begin{titlepage}
%\setcounter{page}{2} - Inclui o n�mero da p�gina
%\thispagestyle{headings}
\vfill

\begin{center}
{\setlength{\unitlength}{1cm}\makebox(12,6.5){\parbox[c]{18cm}{\setlength{\parskip}{0.8cm}\center\vskip -1.2cm\LARGE{\bf Portal de Algoritmos da Universidade de Caxias do Sul \\\large Evolu��o da ferramenta de gerenciamento do portal de algoritmos, visando substituir tecnologias defasadas}\par \normalsize por\par \large Adriano Margarin\par}}}
\end{center}

{\large Trabalho de Conclus�o de Curso submetido ao curso de Bacharelado em Sistemas de Informa��o do Centro de Ci�ncias Exatas e Tecnologia da Universidade de Caxias do Sul, como requisito obrigat�rio para gradua��o.}

\vfill

\begin{center}
{\Large\bf Trabalho de Conclus�o de Curso}
\end{center}

\vfill

\begin{singlespace}
Orientador: {Alexandre Erasmo Krohn Nascimento\par}
Coorientador: {Ricardo Vargas Dorneles\par}

Banca examinadora:\par
\hspace{1cm} {\setlength{\unitlength}{1cm}
\makebox(9,1){\parbox[c]{9cm}{\center Ricardo Vargas Dorneles\\ CCET/UCS}}}\par
\hspace{1cm} {\setlength{\unitlength}{1cm}
\makebox(9,1){\parbox[c]{9cm}{\center Andr� Luis Martinotto\\ CCET/UCS}}}\par

\vfill

\hfill{\setlength{\unitlength}{1cm}\makebox(9,2.5){\parbox[c]{9cm}{\setlength{\parskip}{0.8cm}\center\vskip -1.2cm TRabalho de Conclus�o de Curso apresentado em\\ X de Dezembro de 2015\par Daniel Lu�s Notari\\ Coordenador}}}

\end{singlespace}

\end{titlepage}

\begin{dedicatoria}
\sffamily\itshape

COLOCAR FRASE

\textsc{AUTOR DA FRASE}
\end{dedicatoria}

\begin{agradecimentos}

COLOCAR AGRADECIMENTOS

\vspace{20px}

A todos voc�s, minha sincera gratid�o.

\vspace{90px}

\hfill Adriano Margarin
\end{agradecimentos}

\tableofcontents

\chapter*{Lista de acr�nimos}

\vspace{20px}

\begin{acronym}[XXXXXXXXXX]
\acro{API}[\textit{API}]{\textit{Application Programming Interface}}
\acrodefplural{API}[\textit{APIs}]{\textit{Application Programming Interfaces}}
\acro{HTTP}[\textit{HTTP}]{\textit{Hypertext Transfer Protocol}}
\acro{JAR}[\textit{JAR}]{\textit{Java Archive}}
\acro{JDK}[\textit{JDK}]{\textit{Java Development Kit}}
\acro{AVA}[\textit{AVA}]{\textit{Ambiente Virtual de Aprendizagem}}
\acro{UCS}[\textit{UCS}]{\textit{Universidade de Caxias do Sul}}
\acro{UML}[\textit{UML}]{\textit{Unified Modeling Language}}
\end{acronym}

\listoffigures

\listoftables

\lstlistoflistings


\begin{abstract}

COLOCAR RESUMO

\end{abstract}

\begin{englishabstract}{COLOCAR EM INGL�S O T�TULO DO TCC}

COLOCAR RESUMO EM INGL�S

\end{englishabstract}

\acresetall

\chapter{Introdu��o}\label{cpIntroducao}

No presente trabalho ser� realizada a evolu��o do m�dulo de gerenciamento do portal de algoritmos da Universidade de Caxias do Sul. Nesse mesmo per�odo foi desenvolvido pelo aluno Gabriel Weber um trabalho que visa a evolu��o do analisador algor�tmico \cite{Weber2015}.

O portal de algoritmos, desenvolvido no ano de 2009 pelos professores Ricardo Vargas Dorneles e Delcino Picinin Junior, da Universidade de Caxias do Sul, tem por objetivo auxiliar no ensino da l�gica de programa��o atrav�s da linguagem do portugu�s estruturado, tamb�m conhecida como portugol e � utilizado pelos alunos do \ac{CCET} e p�blico em geral. Esse portal oferece ao aluno a possibilidade de exercitar sua l�gica atrav�s de exerc�cios cadastrados pelos professores, utilizando a linguagem portugol em um editor espec�fico. O aluno submete solu��es de problemas a fim de valid�-las e pode acompanhar seu desempenho atrav�s de um ranking de submiss�es de solu��es corretas. O portal possui uma se��o de gerenciamento que somente administradores podem acessar. Nessa administra��o h� a possibilidade de acompanhar a evolu��o dos alunos, suas submiss�es de solu��es, visualiza��o e edi��o de usu�rio, de problemas, de dados de testes e palavras-chave \cite{Dorneles2004}.

No ano de 2016 o \textit{software} cliente do portal de algor�tmos foi refeito, pois naquele ano foi preciso migrar o programa escrito em \textit{Java Applet} para um aplicativo \textit{Desktop}, trocando informa��es com o portal servido atrav�s de \textit{Web Services}. No entandto, o \textit{software} servidor n�o foi migrado, e � neste contexto que este trabalho est� inserido.

Na cria��o de um problema, o administrador informa um nome e uma descri��o do problema a ser solucionado, dicas e palavras-chave, sendo que as �ltimas  informa��es n�o s�o obrigat�rias. Tamb�m s�o informadas entradas de dados para testes e as sa�das esperadas para as entradas informadas. J� na edi��o do problema, as mesmas informa��es citadas acimas podem ser alteradas.

Outra funcionalidade do portal � de poder acompanhar as submiss�es dos alunos. Selecionando um problema, pode-se visualizar os alunos que submeteram uma solu��o para o mesmo e a partir do aluno � poss�vel ver as solu��es dele e em seguida visualizar as solu��es em si. Essa � uma das formas de acompanhar a evolu��o do aluno. Outra forma � partindo de um determinado aluno e visualizando todas as solu��es que ele j� submeteu.

No m�dulo de administra��o � poss�vel manter todas as informa��es dos usu�rios, problemas, palavras-chave, dados de testes, dentre outras informa��es.

Devido � constante evolu��o tecnol�gica, o portal de algoritmos ficou defasado, com problemas de compatibilidade com os navegadores atuais e pouca ou nenhuma seguran�a. Funcionalidades limitadas no seu gerenciamento tamb�m foram apontadas pelos usu�rios como alvo de melhorias.

As tecnologias utilizadas no desenvolvimento do portal atual, \textit{Python } vers�o 2.6, \textit{Django} vers�o 1.2, \textit{Plugins} \ac{NPAPI} (\textit{Java Applet}), est�o desatualizadas ou foram descontinuadas. No caso da tecnologia \ac{NPAPI}, esta foi totalmente desativada \cite{Python2015, Django2015, NPAPI2015}.

Um dos problemas citados acima s�o as vers�es do \textit{Python} e \textit{Django}, que est�o em vers�es desatualizadas e sem suporte t�cnico por seus desenvolvedores.

O editor de solu��es do portal atual foi programado como sendo um \textit{Applet Java}. \textit{Applets} executam nos navegadores, utilizando a tecnologia de plugins \textit{NPAPI} \cite{NPAPI2015}. \textit{Plugins} \ac{NPAPI} deixaram de ser suportados pelos navegadores atuais, pelo fato de causarem riscos de seguran�a para quem esteja utilizando. Desde o dia 1� de Setembro de 2015, o navegador \textit{Google Chrome} deixou de suportar todas as tecnologias que utilizam \ac{NPAPI}, como \textit{Flash}, \textit{Java}, entre outros. Mesmo eles n�o sendo mais suportados nativamente, � poss�vel ativar isso no navegador, mas isso pode causar vunerabilidades para quem deseja fazer isto \cite{NPAPI2015}.

Para resolver os problems citados acima, este trabalho visa realizar a engenharia reversa do aplicativo, da modelagem de banco de dados atual, reengenharia de \textit{software} e evolu��o de \textit{software} atrav�s de t�cnicas de engenharia de \textit{software} e desenvolvimento.

Atrav�s da engenharia reversa, o programa � analisado e s�o extra�das as informa��es, facilitando a documenta��o de sua organiza��o e funcionalidades \cite{Sommerville2011}.

Para realizar a engenharia reversa � preciso fazer a tradu��o de seu c�digo fonte, sendo que o atual \textit{software} foi desenvolvido utilizando a linguagem de programa��o \textit{Python} e o \textit{Django} \cite{Django2015}.

Com \textit{Python} e \textit{Django} foram desenvolvidas todos os modelos de classes de dom�nio, que ser�o detalhadas no Cap�tulo \ref{cpReengenhariaSoftware}, usando o \ac{ORM} do \textit{Django} foram criadas as tabelas de banco de dados e s�o realizadas as consultas. 

Ser� realizada a diagrama��o das classes atuais utilizando-se da nota��o \ac{UML}. Ser�o utilizados diagramas de classes de dom�nio, que representar�o essas classes, interfaces e suas associa��es, para que depois sejam usadas no desenvolvimento de um modelo de sistema orientado a objetos \cite{Pressman2011, Sommerville2011}.

Atrav�s da engenharia de \textit{software} ser� produzido um novo portal de algoritmos, desde os est�gios inicias da especifica��o do sistema at� sua manuten��o. Com o uso de engenharia de \textit{software} espera-se obter resultados de qualidade e requeridos dentro do cronograma \cite{Sommerville2011}.



Para atingir o objetivo proposto, o trabalho est� organizado da seguinte forma:

No Cap�tulo \ref{cpReferencial} s�o apresentados todos os conceitos metodol�gicos da engenharia de \textit{software}, quais tecnologias que ser�o utilizadas e suas fun��es no contexto do trabalho.

No Cap�tulo \ref{cpReengenhariaSoftware} � apresentada a reengenharia de \textit{software} realizada no portal de algoritmos atual.

No Cap�tulo \ref{cpProposta} � apresentada a modelagem para a evolu��o do \textit{software}, suas interfaces e diagramas relacionados.

No Cap�tulo \ref{cpConsideracoesParciais} s�o apresentadas as considera��es parciais do trabalho.

\chapter{Evolu��o de Software}\label{cpReferencial}

Para manter um \textit{software} �til ele deve mudar continuamente. Essa mudan�a pode ser a partir de uma press�o constante de mudan�as que os usu�rios imp�em, para facilitar ou automatizar algumas tarefas do dia-a-dia.

Todos os \textit{softwares} passar�o pelo processo de envelhecimento, isso � inevit�vel. Algumas causas de problemas podem ser previstas, minimizando os impactos dos danos causados. A continuidade de uso do \textit{software} implica que ocorram mudan�as, que podem ocorrer em regras de neg�cio ou nas expectativas dos usu�rios \cite{Sommerville2011}.

\citet{Rezende2005}, define que um \textit{software} tem um ciclo de vida de no m�ximo 10 anos, quando ele n�o sofre novas implementa��es. O ciclo de vida natural de um \textit{software} abrange as seguintes fases: concep��o, constru��o, implementa��es, implanta��o, maturidade e utiliza��o plena, decl�nio, manuten��o e morte.

Devido a esse ciclo de vida, uma evolu��o de \textit{software} pode ser desencadeada por necessidades de novos componentes, por defeitos relatados ou devido a mudan�as de outros sistemas \cite{Sommerville2011}.

A evolu��o de \textit{software} compreende as mudan�as que ir�o ocorrer a fim de deix�-lo completo e, se poss�vel, livre de erros \cite{Sommerville2011}. Mas para essa evolu��o acontecer � necess�rio considerar diversos fatores que servir�o de base para que um novo software seja constru�do, com base nos requisitos do atual.

O processo de evolu��o varia conforme o tipo de \textit{software} que esteja sendo mantido, dos processos de desenvolvimento e as habilidades das pessoas envolvidas. Em alguns casos a evolu��o pode ser um processo informal, em que na maioria das vezes as mudan�as resultam de conversas com usu�rios. J� em outros casos � um processo formal, envolvendo documenta��o estruturada que � produzida em cada est�gio do processo \cite{Sommerville2011}.

O processo de evolu��o de \textit{software} envolve a compreens�o do \textit{software} que tem que ser alterado. Para tornar-se poss�vel a evolu��o uma das t�cnicas que podem ser usada � a reengenharia no \textit{software} atual, visando melhorar sua estrutura e inteligibilidade \cite{Sommerville2011}.

Para tornar poss�vel a evolu��o de \textit{software} � preciso seguir alguns processos. Nas pr�ximas se��es ser�o apresentadas as metodologias e tecnologias que ser�o utilizadas neste trabalho.

\begin{itemize}
	\item Reengenharia de \textit{Software}
	\item Engenharia Reversa
	\item Engenharia de \textit{Software}
	\item Processo de \textit{Software}
	\item Engenharia de Requisitos
	\item Casos de Uso
	\item Modelagem de Dom�nio
	\item Metodologia ICONIX
	\item Projeto de Arquitetura
	\item Usabilidade
	\item Tecnologias
		\begin{itemize}
			\item Python e Django
			\item Java EE
			\item Wildfly
			\item Padr�es de Projeto: \ac{DAO}
			\item \ac{REST}
			\item AngularJS
		\end{itemize}
\end{itemize}

\section{Reengenharia de Software}

A reengenharia de \textit{software} pode envolver a redocumenta��o do sistema, a refatora��o da arquitetura, a mudan�a de linguagem de programa��o para uma liguagem mais moderna e modifica��es e atualiza��o de estrutura e dos dados de sistema. A funcionalidade n�o � alterada, e geralmente deve evitar grandes mudan�as na arquitetura \cite{Sommerville2011}.

Alguns benef�cios importantes na reengenharia � o risco reduzido quando trata-se de um \textit{software} cr�tico de neg�cio, onde podem haver erros nas especifica��es e atrasos no in�cio do novo, e o custo reduzido, onde o custo da reengenharia se torna significamente menor do que o desenvolvimento de um novo.

A Figura \ref{imgReengenhariaSoftware} demonstra o processo geral da reengenharia, onde a entrada � um sistema legado e a sa�da � uma vers�o melhorada do mesmo.

\FloatBarrier
\begin{figure}[!htb]
	\centering
	\caption{Processo de Reengenharia}
	\includegraphics[width=15cm]{imagens/processo-reengenharia.png}
	\label{imgReengenhariaSoftware}
	Fonte: \cite{Sommerville2011}
\end{figure}
\FloatBarrier

\begin{enumerate}
	\item Tradu��o de c�digo-fonte: atrav�s de alguma ferramenta de tradu��o, o programa � convertido para uma vers�o mais atual da linguagem ou para outra diferente.
	\item Engenharia reversa: o programa � analisado e as informa��es s�o extra�das a partir dele.
	\item Melhoria na estrutura de programa: a estrutura de controle � analisada e modificada para que se torne mais f�cil de ler e entender.
	\item Modulariza��o de programa: partes relacionadas do programa s�o agrupadas, e onde houver redund�ncia, se apropriado, esta � removida. Em alguns casos, esse est�gio pode envolver refatora��o de arquitetura.
	\item Reengenharia dos dados: os dados processados pelo programa s�o alterados para refletir as mudan�as de programa.
\end{enumerate}

Nem sempre � necess�rio seguir todas as etapas da Figura \ref{imgReengenhariaSoftware}. Pode haver casos em que se utiliza o mesmo ambiente de desenvolvimento da linguagem de programa��o. Nesse caso n�o � necess�rio a tradu��o do c�digo \cite{Sommerville2011}.

Na reengenharia um dos processos � a engenharia reversa. Na pr�xima se��o � descrito como ela � utilizada no processo de evolu��o.

\section{Engenharia Reversa}

A engenharia reversa, segundo \citet{Sommerville2011}, consiste em uma t�cnica de an�lise de software com o objetivo de recuperar o projeto e suas especifica��es t�cnicas.

� poss�vel fazer a engenharia reversa atrav�s de diversas formas, na maioria das vezes utilizando os c�digos fontes, al�m dos conhecimentos t�cnicos e experi�ncias dos pr�prios desenvolvedores.

Na se��o seguinte � descrita a Engenharia de \textit{software} e suas respectivas camadas.

\section{Engenharia de Software}

Engenharia de \textit{software} � uma disciplina cujo foco est� em todos os aspectos da produ��o de software, partindo dos est�gios iniciais da especifica��o do \textit{software} at� sua manuten��o, quando o \textit{software} j� est� em funcionamento \cite{Sommerville2011}. De acordo com \citet{Rezende2005}, ``� a metodologia de desenvolvimento e manuten��o de sistemas modulares, com as as seguintes caracter�sticas: processo din�mico, integrado e inteligente de solu��es tecnol�gicas; adequa��o aos requisitos funcionais do neg�cio do cliente e seus respectivos procedimentos pertinentes; efetiva��o de padr�es de qualidade, produtividade e efetividade em suas atividades e produtos; fundamenta��o da Tecnologia da Informa��o dispon�vel, vi�vel, oportuna e personalizada; planejamento e gest�o de atividades, recursos, custos e datas".

Conforme podemos ver na Figura \ref{imgCamadasEngenhariaSoftware}, a engenharia de \textit{software} � uma tecnologia em camadas. A base para a engenharia de \textit{software} � a camada de processos. O processo de engenharia de \textit{software} � o m�todo que permite manter as camadas de tecnologia coesa e possibilita o desenvolvimento do \textit{software} \cite{Pressman2011}.

\FloatBarrier
\begin{figure}[!htb]
	\centering
	\caption{Camadas da Engenharia de Software}
	\includegraphics[width=15cm]{imagens/camadas-engenharia-de-software.png}
	\label{imgCamadasEngenhariaSoftware}
	Fonte: \cite{Pressman2011}
\end{figure}
\FloatBarrier

A engenharia de \textit{software} � realizada atrav�s de processos de \textit{software}, que ser�o descritos a seguir.

\section{Processo de Software}

Um processo de \textit{software} � um conjunto de atividades, a��es e tarefas relacionadas que levam � produ��o de um produto de \textit{software} \cite{Sommerville2011, Pressman2011}. No contexto da engenharia de \textit{software}, um processo n�o � uma prescri��o r�gida de como desenvolver, ele � adapt�vel, que possibilita �s pessoas realizar o trabalho de selecionar e escolher o conjunto apropriado de a��es e tarefas \cite{Pressman2011}.

Dentre muitos processos de \textit{software} existentes todos devem incluir quatro atividades fundamentais \cite{Sommerville2011}.

\begin{itemize}
	\item Especifica��o de \textit{software}
	\item Projeto e implementa��o de \textit{software}
	\item Evolu��o de \textit{software}
\end{itemize}

De acordo com \citet{Sommerville2011}, essas atividades fazem parte do processo de \textit{software}. Na pr�tica eles s�o complexos, possuem subatividades, entre elas levantamento de requisitos, projeto de arquitetura, testes etc.

Para melhor entendimento desses processos, nas pr�ximas se��es ser�o descritas com mais detalhes algumas dessas atividades.

\section{Engenharia de Requisitos}

Engenharia de requisitos de sistemas basicamente � o conjunto das descri��es do que o sistema deve fazer, o que ele oferece de servi�o e restri��es a seu funcionamento \citet{Sommerville2011}. A engenharia de requisitos abrange sete tarefas distintas: concep��o, levantamento, elabora��o, negocia��o, especifica��o, valida��o e gest�o, onde geralmente algumas ocorrem em paralelo e todas podem ser adaptadas � necessidade de cada projeto \cite{Pressman2011}

Somente descrever os requisitos n�o � suficiente, � preciso entender o que est� descrito, e essa � uma das tarefas mais dif�ceis enfrentadas por um engenheiro de \textit{software}.

Os requisitos de \textit{software} frequentemente s�o classificados em funcionais e n�o-funcionais.

Requisitos funcionais s�o declara��es de servi�o que o sistema deve fornecer, de como fornecer, de como o sistema deve reagir a entradas espec�ficas e de como o sistema deve se comportar em determinadas sistua��es. Em alguns casos, os requisitos funcionais tamb�m podem explicitar o que o sistema n�o deve fazer \cite{Sommerville2011}.

Requisitos n�o-funcionais s�o restri��es aos servi�os ou fun��es oferecidas pelo sistema. Incluem restri��es de \textit{timing}, restri��es no processo de desenvolvimento e restri��es impostas pelas normas. Ao contr�rio das caracter�sticas individuais ou servi�os do sistema, os requisitos n�o funcionais muitas vezes aplicam-se ao sistema como um todo \cite{Sommerville2011}.

\section{Casos de Uso}

Casos de uso tem por objetivo descrever os requisitos funcionais, delimita��o do contexto do sistema documentado e entendimento dos requisitos, onde cada caso de uso deve descrever somente uma funcionalidade ou objetivo do sistema \cite{Sommerville2011} e \cite{Pressman2011}.

Um conjunto de casos de uso representa todas as poss�veis intera��es que s�o descritas nos requisitos de sistema. Os atores podem ser pessoas ou outros sistemas e s�o representados como figuras ``palitos" \ e cadas classe de intera��o � representada por uma elipse \cite{Sommerville2011}.

Casos de uso possuem atores e cen�rios, onde os atores podem ser pessoas ou outros sistemas que interagem entre si, e os cen�rios s�o sequ�ncias espec�ficas de a��es. Em outros termos casos de uso � uma cole��o de cen�rios relacionados ao sucesso ou fracasso \cite{Larman2007}.

A Figura \ref{imgCasoUso} apresenta um exemplo de caso de uso de um consult�rio m�dico, onde podemos observar todos os atores envolvidos e suas respectivas a��es.

\FloatBarrier
\begin{figure}[!htb]
	\centering
	\caption{Casos de uso}
	\includegraphics[width=15cm]{imagens/caso-de-uso.png}
	\label{imgCasoUso}
	Fonte: \cite{Sommerville2011}
\end{figure}
\FloatBarrier

A modelagem de caso de uso � um apoiador para a elicita��o de requisitos, geralmente descreve o que o usu�rio espera do sistema. Cada caso de uso representa uma tarefa que envolve a intera��o externa com o sistema \cite{Sommerville2011}.

\section{Metodologia ICONIX}

Para desenvolver um projeto, � necess�rio uma metodologia. Nesse trabalho, ser� utilizada a metodologia ICONIX.

A metodologia ICONIX foi elaborada por Doug Rosenberg e Kendal Scott, a partir de um processo simples e unificado dos pesquisadores Booch, Rumbaugh e Jacobson \cite{Rosenberg2005}.

As vantagens de se utilizar a metodologia ICONIX s�o: metodologia pr�tica, simples, espec�fica de forma objetiva e possui rastreabilidade dos requisitos \cite{Rosenberg2005}.

A metodologia ICONIX utiliza-se de um subconjunto da \ac{UML} no qual apenas 4 diagramas s�o utilizados: diagramas de classe, diagrama de sequ�ncia, diagrama de robustez e caso de usos \cite{Rosenberg2005}.

\subsection{Diagramas de Classe}

Os diagramas de classes s�o usados no desenvolvimento de um modelo de sistema orientado a objetos para mostrar as classes de um sistema e as associa��es entre essas classes \cite{Sommerville2011}.

A Figura \ref{imgExDiagramaClasse} indica as rela��es entre os objetos da classe Paciente e objetos de outras classes \cite{Sommerville2011}.

\FloatBarrier
\begin{figure}[!htb]
	\centering
	\caption{Diagramas de Classe}
	\includegraphics[width=13cm]{imagens/exemplo-diagrama-de-classe.png}
	\label{imgExDiagramaClasse}
	\newline
	Fonte: \cite{Sommerville2011}
\end{figure}
\FloatBarrier

\subsection{Diagrama de Sequ�ncia}

Os diagramas de sequ�ncia geralmente s�o utilizados para modelar as intera��es entre os atores e os objetos em um sistema \cite{Sommerville2011}.

\FloatBarrier
\begin{figure}[!htb]
	\centering
	\caption{Diagrama de Sequ�ncia}
	\includegraphics[width=15cm]{imagens/exemplo-diagrama-sequencia.png}
	\label{imgExSequencia}
	Fonte: \cite{Sommerville2011}
\end{figure}
\FloatBarrier

A Figura \ref{imgExSequencia} pode ser lida da seguinte maneira:

\begin{enumerate}
	\item A recepcionista do m�dico aciona o m�todo VerInfo em uma inst�ncia P da classe de objeto InfoPaciente, fornecendo o identificador do paciente (PID, do ingl�s \textit{patient's identifier}). A inst�ncia P � um objeto de interface  do usu�rio, exibido como um formul�rio  que mostra os dados do paciente.
	\item A inst�ncia P chama o banco de dados para retornar as informa��es necess�rias, fornecendo o identificador da recepcionista, que permite a verifica��o de prote��o (nessa fase, n�o importa de onde vem o esse UID - do ingl�s, \textit{user's identifier}).
	\item O banco de dados verifica, com o sistema de autoriza��o, que o usu�rio est� autorizado a essa a��o.
	\item Se autorizado, as informa��es de pacientes s�o retornadas, e um formul�rio � preenchido na tela do usu�rio. Se falhar a autoriza��o, aparece uma mensagem de erro.
\end{enumerate}

\subsection{Diagrama de Robustez}

Este � um diagrama que n�o existe na \textit{UML} e � geralmente um diagrama de colabora��o adaptado e que faz uso dos estere�tipos \textit{entity}, \textit{boundary} e \textit{control}. Ele � utilizado em processos como o ICONIX para passar da an�lise (o que) para o desenho (como). Esse � um diagrama que n�o � necess�rio ser mantido atualizado, uma vez que � utilizado apenas para a transi��o entre os \textit{softwares}. A an�lise de robustez consiste ent�o em ler o texto do caso de uso e identificar de forma preliminar, o conjunto de objetos que ir�o participar do caso de uso.

A Figura \ref{imgExRobustez} representa a intera��o entre o usu�rio e as interface de um sistema, bem como todas as intera��es entre as interfaces.

Como podemos observar na Figura \ref{imgExRobustez}, o ator usu�rio clica no ``�cone de contatos" na tela principal, ap�s o clique � exibido a tela de contatos, na sequ�ncia o ator clica em ``adicionar novo" no qual resulta na exibi��o da tela de novo contato, continuando a a��o o ator preenche os campos selecionados e faz a a��o de salvar contato na mem�ria retornando assim para para tela de exibi��o de contatos.

\FloatBarrier
\begin{figure}[!htb]
	\centering
	\caption{Diagrama e Robustez}
	\includegraphics[width=10cm]{imagens/exemplo-diagrama-robustez.png}
	\label{imgExRobustez}
	\newline
	Fonte: \cite{Galeote2015}
\end{figure}
\FloatBarrier

\subsection{Casos de Usos}

Diagrama de casos de usos descrevem funcionalidades propostas para o novo sistema, fornecendo uma descri��o clara e consistente do que o sistema deve fazer.

A Figura \ref{imgExCasoUso} representa o caso de uso de transfer�ncia de dados que envolve os atores Recepcionista do m�dico e Sistema de registro de pacientes \cite{Sommerville2011}.

Como podemos observar na Figura \ref{imgExCasoUso} a recepcionista do m�dico realiza a transfer�ncia de dados para o sistema de registro de pacientes.

\FloatBarrier
\begin{figure}[!htb]
	\centering
	\caption{Casos de uso de transfer�ncia de dados}
	\includegraphics[width=13cm]{imagens/exemplo-caso-de-uso.png}
	\label{imgExCasoUso}
	\newline
	Fonte: \cite{Sommerville2011}
\end{figure}
\FloatBarrier

\section{Modelos de Dom�nio}

Um modelo de dom�nio exibe como est� organizado o sistema em termos de seus componentes e seus relacionamentos. Podem ser est�ticos ou din�micos, onde os modelos est�ticos mostram a estrutura do sistema e os din�micos, onde � exibido quando ele est� em execu��o \cite{Sommerville2011}.

De acordo com \citet{Sommerville2011}, ``os diagramas de classe s�o utilizados no desenvolvimento de um modelo de sistema orientado a objetos para mostrar as classes de um sistema e as associa��es entre essas classes".

Um modelo de dom�nio � uma representa��o visual de classes conceituais, ou objetos do mundo real, em um dom�nio \cite{Larman2007}. Tamb�m s�o conhecidos como modelos conceituais.

\section{Projeto de Arquitetura}

O projeto de arquitetura � a representa��o da estrutura de dados e seus componentes. Ele compreende como o sistema deve ser organizado a fim de atender as necessidades levantadas na engenharia de requisitos \cite{Sommerville2011, Pressman2011}.

Na arquitetura em camadas o \textit{software} � dividido em subconjuntos funcionais denomidadas camadas, onde cada parte possui um pr�posito bem definido e cada parte conhece apenas a parte imediatamente inferior \cite{Sommerville2011}.

Na arquitetura em camadas encontram-se todas as partes do \textit{software}, e define-se a responsabilidade de cada uma. Esse padr�o de arquitetura � uma das maneiras de se conseguir independ�ncia entre elas, como por exemplo o padr�o \ac{MVC}, em que s�o separadas as camadas de apresenta��o da intera��o dos dados do sistema \cite{Sommerville2011, Pressman2011}.

\section{Usabilidade}

Sistemas devem ser flex�veis, simples e agrad�veis de usar. A usabilidade � a principal ci�ncia da \ac{IHC}, \ac{IHC} tem por objetivo produzir sistemas us�veis, seguros e funcionais.

Na \ac{IHC}, a usabilidade se refere a simplicidade e facilidade com que uma interface de um sistema pode ser utilizado. A import�ncia do \ac{IHC} no desenvolvimento de \textit{software} � de ter uma defini��o de padr�o visual, padr�o de mensagens e prototipa��o e valida��o de telas com usu�rio, medindo a usabilidade e garantindo a padroniza��o e consist�ncia.

De acordo com \citet{Benyon2011}, um sistema com usabilidade ter� as seguintes caracter�sticas:

\begin{itemize}
	\item Ser� eficiente no sentido de que as pessoas poder�o fazer as coisas mediante uma quantidade adequada de esfor�o.
	\item Ser� eficaz no sentido de que conter� as fun��es e o conte�do de informa��es adequadas e organizadas de forma apropriada.
	\item Ser� f�cil aprender como fazer as coisas e ser� f�cil de lembrar como faz�-las ap�s algum tempo.
	\item Ser� seguro de operar na variedade de contextos em que ser� usado.
	\item Ter� um alto grau de utilidade no sentido de que far� as coisas que as pessoas querem que sejam feitas.
\end{itemize}

O portal de algoritmos atual apresenta algumas telas confusas, que podem ser observadas no Cap�tulo \ref{cpProposta} na Se��o \ref{scPrototipos}. Esse trabalho pretende melhor�-las.

At� aqui foram apresentadas as metodologias que ser�o utilizadas na evolu��o do aplicativo. Na se��o seguinte ser�o apresentadas as tecnologias escolhidas para a evolu��o do gerenciamento do portal de algoritmos.

\section{Tecnologias}

Nesta se��o ser�o descritas as tecnologias que v�o ser utilizadas na evolu��o do portal de algoritmos, tais como a linguagem de programa��o \textit{Java}, \ac{REST} e \textit{AngularJS}. S�o tecnologias bem consolidades no mercado, com \textit{upgrade} garantido por tempo indeterminado, mantidas por empresas conhecidas e de grande porte.

\subsection{Python e Django}

O \textit{Python} � uma linguagem interpretada de alto n�vel, criada por Guido Van Rossum em 1989 e lan�ada em 1991 e atualmente possui o modelo de desenvolvimento comunit�rio \cite{Python2015}. Com o aux�lio do \textit{framework Django}, criado originalmente para gerenciar conte�dos de um jornal da cidade de Lawrence, no Kansas, � poss�vel definir a modelagem de dados atrav�s de classes \textit{Python} e gerar tabelas do banco de dados para manipula��o sem a necessidade direta de \ac{SQL} \cite{Django2015}.

\subsection{Java EE}

A linguagem Java � uma linguagem de programa��o orientada a objetos, com portabilidade, independ�ncia de plataforma, extensas bibliotecas de rotinas que facilitam recursos de rede e seguran�a, podendo executar programas via rede com restri��es de execu��es \cite{Java2015}.

Al�m disso, ela se destaca com a similaridade de sintaxe da linguagem C/C++, facilidade de internacionaliza��o, simplicidade nas especifica��es, entre outras \cite{Java2015}.

O \ac{JAVA EE} � uma s�rie de especifica��es que descrevem como deve ser implementado um \textit{software} que faz uso de servi�os de infraestrutura. Tamb�m � considerado uma maneira de desenvolver aplicativos com suporte a escabilidade, flexibilidade e seguran�a \cite{Java2015}.

\subsection{WildFly}

O servidor de aplica��o \textit{WildFly} implementa a mais recente vers�o do \ac{JAVA EE}, sendo mantido pela \textit{Red Hat} \cite{Wildfly2015}.

Os \textit{frameworks} que comp�em o \ac{JAVA EE} s�o fortemente testados em diversas combina��es. De acordo com padr�es com os quais o servidor foi desenvolvido o desenvolvedor pode focar nas regras de neg�cio e utilizar-se dos recursos de infraestrutura fornecidas pelo \textit{framework} \cite{Wildfly2015}.

\subsection{DAO}

\ac{DAO} � um padr�o de projeto para trabalhar com fontes de dados, que podem ser um banco de dados relacional, banco de dados orientado a objetos, entre outros ... \cite{Deepak2004}.

Com \ac{DAO} � poss�vel adaptar a diferentes esquemas de armazenamento sem afetar outros componentes de neg�cio, basicamente o \ac{DAO} atua como um adaptador entre o componente de apresenta��o de dados e a fonte de dados \cite{Deepak2004}.

\subsection{REST}

A \ac{REST} � um estilo de arquitetura que define um conjunto de restri��es e propriedades baseado no \ac{HTTP}, utilizando-se dos verbos desse protocolo. As princ�pios fundamentais do \ac{REST} s�o: d� a todas as coisas um identificador, vincule as coisas, utilize m�todos padronizados, recursos com m�ltiplas representa��es e comunique sem estado \cite{Rest2000}.

O \ac{REST} possui um conjunto de opera��es bem definidas, os mais importantes s�o \textit{GET}, \textit{POST}, \textit{PUT} e \textit{DELETE} \cite{Restful2013}. Conforme \cite{Rest2000}, \ac{REST} � um modelo de arquitetura bem definido para servir aplica��es \textit{WEB}.

\subsection{AngularJS}

\textit{AngularJS} foi criado por Misko Hevery e Adam Abrons em 2009, sendo seu c�digo fonte aberto (\textit{Open Source}). Ele � um \textit{framework JavaScript} que � executado no navegador de internet do usu�rio, atrav�s do qual � poss�vel aumentar sua produtividade no desenvolvimento \textit{WEB} \cite{Angular2014}.

\textit{AngularJs} foi constru�do com a cren�a de que a programa��o declarativa � a melhor escolha para a constru��o de intefaces de usu�rios. Para isso, o \textit{AngularJs} aumenta o vocabul�rio do \ac{HTML} padr�o, tornando mais vers�til o desenvolvimento de sistemas \ac{WEB} \cite{Angular2014}.

O resultado � o desenvolvimento reutiliz�vel e aplica��o sustent�vel de componentes, deixando para tr�s c�digos desnecess�rios e mantendo a equipe focada no que � importante \cite{Angular2014}.

O padr�o \ac{MVC} ganhou muita popularidade nas f�bricas de \textit{software}, tornando-se um dos projetos de arquitetura empresarial mais utilizados. Basicamente o modelo (\textit{Model}) consiste nos dados da aplica��o, regras de neg�cios, l�gicas e fun��es. A vis�o (\textit{View}) � a sa�da de representa��o dos dados e o controle (\textit{Controller}) faz a intermedia��o da entrada ou sa�da para o modelo ou vis�o.

Uma aplica��o em \textit{AngularJS} trabalha com \ac{HTML} e \ac{MVC}, mas tamb�m possui servi�os, diretivas e filtros \cite{Angular2014}.

A \textit{View} � escrita em \ac{HTML}, que faz com que \textit{web designers} e programadores trabalhem lado a lado, com a ajuda das diretivas, que s�o um tipo de extens�o do vocabul�rio \ac{HTML}, que traz a capacidade de executar tarefas de linguagem de programa��o \cite{Angular2014}.

Atr�s da \textit{View} existe um \textit{Controller}, que cont�m toda a l�gica do neg�cio usado pela \textit{View}.

A conex�o entre a vis�o e o controlador � feita por um objeto compartilhado
chamado \textit{scope}. Ele est� localizado entre eles e � usado para trocar informa��es relacionados com o \textit{Model}.

\newpage

A Figura \ref{imgAngularJS} representa a intera��o entre os componentes do \textit{AngularJS}

\FloatBarrier
\begin{figure}[!htb]
	\centering
	\caption{Intera��o entre AngularJS e Arquitetura}
	\includegraphics[width=15cm]{imagens/diagrama-angularjs.png}
	\label{imgAngularJS}
	Fonte: \cite{Angular2014}
\end{figure}
\FloatBarrier

\section{Ambiente Virtual de Aprendizagem}

\ac{AVA} � um aplicativo \ac{WEB} que possui um conjunto de elementos tecnol�gicos, onde s�o disponibilizadas ferramentas que permitem o acesso a um ou mais cursos ou disciplinas de uma institui��o de ensino. De modo geral, um AVA refere-se ao uso de recurso digitais de comunica��o, principalmente, atrav�s de softwares educacionais via internet que re�nem diversas ferramentas de intera��o \cite{Oliveira2004, Valentini2005}.

O objetivo de um ambiente virtual de aprendizagem � de facilitar o acesso de alunos ao ensino, pr�ticas de exerc�cios e livros online para consulta. Na Universidade de Caxias do Sul o \ac{AVA} j� � utilizado desde meados de 2005, onde � poss�vel acessar os materiais disponibilizados pelos professores em suas respectivas disciplinas, podendo tamb�m acompanhar o cronograma, entre outras funcionalidades \cite{Oliveira2004, Valentini2005}.

O portal de algoritmos � um ambiente virtual de aprendizagem utilizado pelos alunos da \ac{UCS} nas disciplinas de ci�ncias exatas. O portal tem por objetivo auxiliar no ensino da l�gica de programa��o atrav�s da linguagem do portugu�s estruturado \cite{Dorneles2004}.

Vimos at� aqui todos os conceitos necess�rios para o desenvolvimento do trabalho, no cap�tulo a seguir veremos a reengenharia do portal de algoritmos atual, onde s�o descritos os problemas do \textit{software} e modelagem de uma nova aplica��o.

 descrita a modelagem e seus problemas de usabilidade e de tecnologia.

\chapter{Reengenharia de Software do Portal de Algoritmos da UCS}\label{cpReengenhariaSoftware}

Nesse cap�tulo � descrita a situa��o atual do sistema, sua modelagem e arquitetura.

\section{Diagrama de Classe de Dom�nio}

O diagrama de classe de dom�nio � a representa��o visual de classes conceituais, ou objetos do mundo real. Tamb�m � chamado de modelo conceitual, que significa uma representa��o de classes conceitos do mundo real, n�o de objetos de \textit{software} \cite{Larman2007}.

Durante a tradu��o do c�digo fonte dos modelos de classes do \textit{Django}, foi constatada nenhuma padroniza��o em nomes de v�riaveis e classes. Abaixo algumas considera��es:

\begin{itemize}
	\item Os campos n�o possuem nomes padronizados utilizando \textit{CamelCase};
	\item Nomes de campos escritos em ingl�s e em portugu�s;
	\item Nomes de classes escritos em ingl�s e em portugu�s;
	\item Nomes de classes n�o s�o intuitivos quanto ao seu objetivo;
\end{itemize}

\textit{CamelCase} � a denomina��o em ingl�s para a pr�tica de escrever palavras compostas, onde cada palavra � iniciada com min�scula ou mai�scula e unidas sem espa�o. Exemplo:

\begin{itemize}
	\item lowerCamelCase;
	\item UpperCamelCase;
\end{itemize}

\newpage

A Figura \ref{imgDominio} representa as classes do portal que � utilizado atualmente.

\FloatBarrier
\begin{figure}[!htb]
	\centering
	\caption{Diagrama de Dom�nio do Portal de Algoritmos Atual}
	\includegraphics[width=15cm]{UML/portal-atual.png}
	\label{imgDominio}
	Fonte: (AUTOR, 2015)
\end{figure}
\FloatBarrier

Abaixo � descrito o que cada classe representa no \textit{software} atual, visa-se manter todas as funcionalidades atuais ap�s a evolu��o do mesmo, e adicionar algumas novas.

\subsection{User}

\textit{User} representa os usu�rios. O usu�rio tem algumas fun��es importantes no sistema. Sendo somente com ele � poss�vel ter acesso a �reas restritas, bem como resolver os exerc�cios do portal.

\subsection{Groups}

\textit{Group} representa os grupos do \ac{ORM} do \textit{framework Django}. Essa classe tem por objetivo agrupar usu�rios com determinadas permiss�es, mas para o sistema atual ela n�o � utilizada.

\subsection{Permission}

\textit{Permission} corresponde �s permiss�es utilizadas no sistema. Tem por objetivo definir permiss�es de acesso ao sistema, como por exemplo se um usu�rio que possui acesso ao gerenciamento do portal esse pode cadastrar um aluno ou um novo problema.

\subsection{ContentType}

\textit{ContentType} corresponde aos tipos de conte�do. � uma classe padr�o do \textit{Framework Django}, que representa informa��es dos modelos utilizados no projeto. Sempre que � criado um modelo novo � criado um tipo de conte�do automaticamente.

\subsection{Aluno}

Aluno corresponde aos alunos cadastrados no sistema. No portal essa classe � um complemento aos dados do usu�rio, representando um aluno com suas respectivas informa��es.

\subsection{Professor}

Professor corresponde aos professores que podem cadastrar problemas no sistema. No portal � um complemento aos dados do usu�rio, representando um professor com suas respectivas informa��es.

\subsection{Turma}

Turma corresponde a turmas cadastradas no sistema. As turmas s�o usadas para agrupar alunos. Essa classe n�o � utilizada no sistema atual.

\subsection{Problema}

Problema corresponde aos problemas. Essa classe representa todas as informa��es correspondentes a um problema. Somente administradores do portal tem permiss�o para gravar infoma��es. Essas permiss�es n�o s�o controladas pela classe de dom�nio \textit{Permission}. Esse controle funciona atrav�s do campo \textit{is\_superuser} da classe de dom�nio \textit{User}.

\subsection{Pchave}

Pchave corresponde �s palavras-chave. Representa uma palavra-chave que � um facilitador para a pesquisa de problemas.

\subsection{PchaveProb}

PchaveProb corresponde � rela��o entre a palavra-chave e um problema. Essa classe � utilizada para poder referenciar mais de uma palavra-chave para o mesmo problema.

\subsection{EProblema}

EProblema corresponde �s entradas e sa�das esperadas de um determinado problema. Nela � posss�vel informar alguma caracter�stica que ajude o usu�rio a resolver um determinado exerc�cio.

\subsection{SProblema}

SProblema corresponde �s solu��es submetidas pelos usu�rios em algum exerc�cio resolvido. Possui rela��o direta com o usu�rio e um problema.

Nesta se��o foram descritas as classes de dom�nios do portal de algoritmos, a seguir � exibida as interfaces gr�ficas.

\section{Interfaces Gr�ficas}

Nessa se��o s�o descritas as interfaces gr�ficas atuais do software, bem como os problemas encontrados nelas.

\subsection{Cadastro de Aluno}

A Figura \ref{imgCadastroAluno} � a interface de Cadastro de  Aluno, que � realizado pelo pr�prio aluno. As informa��es do aluno ser�o mantidas nas intefaces novas, mantendo assim consist�ncia nas informa��es dos mesmos.

\FloatBarrier
\begin{figure}[!htb]
	\centering
	\caption{Cadastro de Aluno}
	\includegraphics[width=15cm]{imagens/portal-antigo/cadastro-aluno.png}
	\label{imgCadastroAluno}
	Fonte: (AUTOR, 2015)
\end{figure}
\FloatBarrier

Problemas encontrados:

\begin{itemize}
	\item \textit{Java Applet} que funciona somente no Internet Explorer ap�s algumas configura��es nas exce��es do \textit{Java}.
	\item Pouca Usabilidade, uma vez que o cadastro do aluno est� na mesma interface da programa��o de algoritmos.
\end{itemize}

\subsection{Cria��o de Solu��o de Problemas}

A Figura \ref{imgProblemaSelecionado} exibe um aluno j� autenticado no portal e um problema e sua solu��o j� selecionados. Nessa interface gr�fica encontramos as seguintes funcionalidades:

\begin{itemize}
	\item Criar nova solu��o.
	\item Salvar solu��o.
	\item Executar solu��o.
	\item Validar solu��o.
\end{itemize}

\FloatBarrier
\begin{figure}[!htb]
	\centering
	\caption{Cria��o de Solu��o de Problemas}
	\includegraphics[width=15cm]{imagens/portal-antigo/problema-selecionado.png}
	\label{imgProblemaSelecionado}
	Fonte: (AUTOR, 2015)
\end{figure}

Problemas encontrados:

\begin{itemize}
	\item \textit{Java Applet} que funciona somente no Internet Explorer ap�s algumas configura��es nas exce��es do \textit{Java}.
	\item Pouca Usabilidade, uma vez que para selecionar outro problema s�o necess�rias diversas confirma��es antes de executar a a��o.
\end{itemize}
\FloatBarrier

\newpage

\subsection{Gerenciamento de Alunos}

A Figura \ref{imgAdministracaodeAlunos} � a interface de gerenciamento de alunos. Nessa interface encontramos as seguintes funcionalidades:

\begin{itemize}
	\item Pesquisa de problemas: � poss�vel realizar uma pesquisa livre, entendendo-se por livre qualquer palavra digitada no campos de pesquisa.
	\item Pesquisa de alunos: � poss�vel realizar uma pesquisa livre, entendendo-se por livre qualquer palavra digitada no campos de pesquisa.
	\item Selecionando um problema, automaticamente o sistema faz uma pesquisa dos alunos que j� solucionaram o problema, e selecionando o aluno � realizada uma pesquisa para encontrar as solu��es desse aluno.
	\item Selecionando um aluno, automaticamente o sistema faz uma pesquisa dos problemas que esse aluno j� resolveu, e selecionando o problema � realizada uma pesquisa para encontrar as solu��es desse aluno.
\end{itemize}

\FloatBarrier
\begin{figure}[!htb]
	\centering
	\caption{Ger�ncia de Alunos}
	\includegraphics[width=15cm]{imagens/portal-antigo/gerencia-de-alunos.png}
	\label{imgAdministracaodeAlunos}
	Fonte: (AUTOR, 2015)
\end{figure}
\FloatBarrier

Problemas encontrados:

\begin{itemize}
	\item Pouca Usabilidade, uma vez que os campos dispostos na interface de maneira pouco intuitiva ao usu�rio.
\end{itemize}

\subsection{Gerenciamento de Problemas}

A Figura \ref{imgAdministracaodeProblemas} � a interface de gerenciamento de problemas. Nessa interface encontramos as seguintes funcionalidades:

\begin{itemize}
	\item Cadastrar novo problema
	\item Editar um problema:
	\begin{itemize}
		\subitem Alterar descri��o e dicas
		\subitem Alterar palavras-chave
		\subitem Alterar entradas e sa�das
	\end{itemize}
\end{itemize}

\FloatBarrier
\begin{figure}[!htb]
	\centering
	\caption{Ger�ncia de Problemas}
	\includegraphics[width=15cm]{imagens/portal-antigo/gerencia-de-problemas.png}
	\label{imgAdministracaodeProblemas}
	Fonte: (AUTOR, 2015)
\end{figure}
\FloatBarrier

Problemas encontrados:

\begin{itemize}
	\item Pouca Usabilidade.
	\begin{itemize}
		\subitem Campos dispostos na interface de maneira pouco intuitiva ao usu�rio.
		\subitem N�o possui campo de pesquisa.
		\subitem A��es descentralizadas que confundem o usu�rio.
	\end{itemize}
\end{itemize}

\newpage

\subsection{Edi��o de Palavra-Chave}

A Figura \ref{imgAdministracaodeProblemasPalavraChave} mostra a mensagem de sucesso ap�s a edi��o de alguma informa��o do problema.

\FloatBarrier
\begin{figure}[!htb]
	\centering
	\caption{Edi��o de palavras-chave}
	\includegraphics[width=15cm]{imagens/portal-antigo/gerencia-de-problemas-editando-palavras-chaves.png}
	\label{imgAdministracaodeProblemasPalavraChave}
	Fonte: (AUTOR, 2015)
\end{figure}
\FloatBarrier

Problemas encontrados:

\begin{itemize}
	\item Pouca Usabilidade: a mensagem n�o possui uma informa��o clara do que foi editado, ou o que foi realizado.
\end{itemize}

\newpage

\subsection{Solu��o de Problemas}

A Figura \ref{imgAdministracaodeProblemasSolucao} exibe a solu��o de um problema em uma nova janela.

\FloatBarrier
\begin{figure}[!htb]
	\centering
	\caption{Visualizando solu��o de um aluno}
	\includegraphics[width=15cm]{imagens/portal-antigo/administracao-de-problemas-visualizando-solucao-partindo-do-problema.png}
	\label{imgAdministracaodeProblemasSolucao}
	Fonte: (AUTOR, 2015)
\end{figure}
\FloatBarrier

Problemas encontrados:

\begin{itemize}
	\item Pouca Usabilidade.
	\begin{itemize}
		\subitem A nova janela que � aberta � muito pequena e sem a possibilidade de aumentar.
		\subitem N�o � poss�vel editar.
	\end{itemize}
\end{itemize}


Os problemas encontrados no \textit{software} atual prejudicam a usabilidade do mesmo. Atualmente ele est� limitado a apenas um navegador de internet, no caso o \textit{Internet Explorer}.

Para solucionar e deix�-lo mais us�vel, fazendo com que mais alunos sejam beneficiados pelo portal, bem como facilitando o uso por parte dos professores,  no pr�ximo cap�tulo ser� descrita toda a modelagem, novas interfaces e novas funcionalidades, utilizando-se de tecnologias mais atuais.


\chapter{Proposta de Solu��o}\label{cpProposta}

O presente trabalho tem por objetivo realizar a evolu��o do gerenciamento do portal de algoritmos. Para tanto, � realizada a engenharia reversa do \textit{software} atual, atrav�s da an�lise do c�digo fonte, suas funcionalidades, sua arquitetura e seu banco de dados.

Para descrever a evolu��o de \textit{software}, nas se��es seguintes ser�o apresentados conceitos e artefatos da engenharia de \textit{software}. Utilizando-se de artefatos da metodologia ICONIX ser� modelado o \textit{software}.

A solu��o proposta ser� desenvolvida na linguagem de programa��o Java, a fim de unificar as tecnologias do gerenciamento do portal de algoritmos com seu analisador algor�tmico.

O \textit{software} ser� constru�do com um arquitetura orientada a servi�os, cujo objetivo � que ele seja disponibilizado em interfaces, ou seja, seja acess�vel atrav�s de \ac{REST}.

Na evolu��o do \textit{software} ser�o mantidas as funcionalidades atuais e adicionadas novas funcionalidas:

\begin{itemize}
	\item Manter Administradores.
	\item Manter Grupos de Administradores.
	\item Manter Usu�rios, Alunos e Professores.
	\item Manter Grupos de Alunos.
	\item Manter Institui��es.
	\item Manter Permiss�es.
	\item Manter Pa�ses, Estados e Cidades.
	\item Manter Tipos de Problemas.
		\subitem Manter Problemas.
		\subitem Manter Entradas e Sa�das
		\subitem Manter Palavras-chave
		\subitem Manter Listas de Problemas
		\subitem Manter Solu��es de Problemas
	\item Chat para comunica��o entre usu�rios
	\item Relat�rios.
		\subitem Relat�rios de acesso.
			\subsubitem Esse relat�rio demonstra todas as vezes que um usu�rio acessou o portal, podendo ser filtrado por data. 
		\subitem Relat�rio de problemas solucionados.
			\subsubitem Esse relat�rio demonstra todas as solu��es submetidas pelos usu�rios, podendo ser filtrado por data.
\end{itemize}

\section{Diagrama de Classe de Dom�nio}

A figura \ref{imgDiagramaNovo} representa o diagrama de classe de dom�nio proposto para a evolu��o do portal de algoritmos novo.

Nesse novo diagrama foram melhoradas as descri��es dos campos, descrevendo-os em ingl�s e utilizando \textit{CamelCase}.

\FloatBarrier
\begin{figure}[!htb]
	\centering
	\caption{Diagrama de Dom�nio do Portal de Algoritmos Novo}
	\includegraphics[width=15cm]{UML/diagrama-portal-novo.png}
	\label{imgDiagramaNovo}
	Fonte: (AUTOR, 2015)
\end{figure}
\FloatBarrier

\subsection{User}

\textit{User} representa os usu�rios. Possui a mesma fun��o da classe \textit{User} do portal de algoritmos antigo.

\subsection{Teacher}

\textit{Teacher} representa os professores. Essa classe � equivalente � classe Professor do portal de algoritmos antigo.

\subsection{Student}

\textit{Student} representa os estudantes. Essa classe � equivalente � classe Aluno do portal de algoritmos antigo.

\subsection{Admins}

\textit{Admins} representa a classe de administradores do portal de algoritmos. Essa classe foi criada para fazer a associa��o entre um \textit{User} e um \textit{GroupAdmin}.

\subsection{GroupAdmin}

\textit{GroupAdmin} representa a classe de Grupo de Administradores. Essa classe � equivalente � classe \textit{Group} do portal de algoritmos antigo.

\subsection{Permission}

\textit{Permission} representa as permiss�es. Essa classe � equivalente � classe \textit{Permission} do portal de algoritmos antigo.

\subsection{Group}

\textit{Group} representa os grupos de alunos. � equivalente � classe Turma no portal de algoritmos antigo, que n�o estava sendo utilizada.

\subsection{Institution}

\textit{Institution} representa as institui��es que utilizam o portal de algoritmos e serve para identificar a localiza��o de um professor. Essa classe surgiu da necessidade de padroniza��o, pois antes era um campo de texto livre na classe Professor.

\subsection{Problem}

\textit{Problem} representa os problemas do portal de algoritmos. � equivalente � classe Problema do portal de algoritmos antigo.

\subsection{ProblemSolution}

\textit{ProblemSolution} representa as solu��es submetidas pelos usu�rios. � equivalente � classe SProblema.

\subsection{ProblemList}

\textit{ProblemList} representa as listas de problemas. � uma classe nova que surgiu da necessidade de criar listas espec�ficas para grupos ou alunos.

\subsection{ProblemKeyWord}

\textit{ProblemKeyWord} representa a rela��o entre palavras-chave com o problema. � equivalente � classe PchaveProb do portal de algoritmos antigo.

\subsection{KeyWord}

\textit{KeyWord} representa as palavras-chave. � equivalente � classe Pchave do portal de algoritmos antigo.

\subsection{TestData}

\textit{TestData} representa os dados de teste dos problemas. � equivalente � classe EProblema.

\subsection{ProblemType}

\textit{ProblemType} representa os tipos de problemas. � uma classe nova que surgiu da necessidade de agrupar os problemas por tipos, n�o mais pela nomenclatura utilizada atualmente.

\subsection{Country}

\textit{Country} representa os pa�ses do portal de algoritmos, classe nova que surgiu da necessidade de padronizar nomes de pa�ses.

\subsection{State}

\textit{State} representa os estados do portal de algoritmos, classe nova que surgiu da necessidade de padronizar nomes de estados.

\subsection{City}

\textit{City} representa as cidades do portal de algoritmos, classe nova que surgiu da necessidade de padronizar nomes de cidades.

Ao realizar a modelagem do diagrama de classes de dom�nios do novo portal de algoritmos, foram padronizadas descri��es de campos e descri��es de classes para o padr�o \textit{CamelCase}.

Na se��o seguinte s�o descritos os requisitos de projeto que s�o os requisitos funcionais e n�o-funcionais.

\section{Requisitos de projeto}

Antes do desenvolvimento do \textit{software} � preciso realizar o levantamento de requisitos funcionais e n�o-funcionais. Esses requisitos descrevem o que o sistema deve fazer e suas restri��es.

\subsection{Requisitos Funcionais}

A tabela \ref{tbRequisitoFuncionais} mostra os requisitos funcionais do portal de algoritmos.

\FloatBarrier
\begin{table}[!htb]
	\centering
	\caption{Requisitos funcionais}
	\label{tbRequisitoFuncionais}
	\includegraphics[width=15cm]{UML/requisitos/1.png}
	Fonte: (AUTOR, 2015)
\end{table}
\FloatBarrier

Entende-se quando se refere em ``manter", listar, cadastrar, editar e deletar um objeto.

\newpage

\subsection{Requisitos N�o-Funcionais}

A tabela \ref{tbRequisitosNaoFuncionais} s�o os requisitos n�o-funcionais do portal de algoritmos.

\FloatBarrier
\begin{table}[!htb]
	\centering
	\caption{Requisitos n�o-funcionais}
	\label{tbRequisitosNaoFuncionais}
	\includegraphics[width=15cm]{UML/requisitos/2.png}
	Fonte: (AUTOR, 2015)
\end{table}
\FloatBarrier

Cada requisito funcional elicitado pode se tornar um caso de uso. Esses casos de uso ser�o descritos na se��o seguinte.

\section{Casos de Uso}

Ap�s o levantamento e descri��o dos requisitos do \textit{software}, pode ser criado o diagrama macro de intera��o entre o usu�rio e os casos de uso do sistema, ressaltando que um \textit{software} dessa natureza deve ter um ambiente de execu��o com acesso a \textit{internet}.

\newpage 

Na figura \ref{imgCasoUsoAdministrador} est�o representados os casos de uso em que o ator ``Administrador" \ est� envolvido para o gerenciamento do portal.

\FloatBarrier
\begin{figure}[!htb]
	\centering
	\caption{Casos de Uso que envolvem o ator Administrador}
	\includegraphics[width=15cm]{UML/casos-de-uso-administrador.jpg}
	\label{imgCasoUsoAdministrador}
	Fonte: (AUTOR, 2015)
\end{figure}
\FloatBarrier

Na figura \ref{imgCasoUsoProfessor} est�o representados os casos de uso em que o ator ``Professor" \ est� envolvido para o gerenciamento do portal.

\FloatBarrier
\begin{figure}[!htb]
	\centering
	\caption{Casos de Uso que envolvem o ator Professor}
	\includegraphics[width=8cm]{UML/casos-de-uso-professor.jpg}
	\label{imgCasoUsoProfessor}
	\\
	Fonte: (AUTOR, 2015)
\end{figure}
\FloatBarrier

Na figura \ref{imgCasoUsoAluno} est�o representados os casos de uso em que o ator ``Aluno" \ est� envolvido para o gerenciamento do portal.

\FloatBarrier
\begin{figure}[!htb]
	\centering
	\caption{Casos de Uso que envolvem o ator Aluno}
	\includegraphics[width=8cm]{UML/casos-de-uso-aluno.jpg}
	\label{imgCasoUsoAluno}
	\\
	Fonte: (AUTOR, 2015)
\end{figure}
\FloatBarrier

A seguir s�o apresentados detalhadamente cada um dos caso de uso apresentados nas figuras \ref{imgCasoUsoAdministrador}, \ref{imgCasoUsoProfessor} e \ref{imgCasoUsoAluno}. Ser�o apresentados os fluxos de execu��o principais e alternativos de cada caso de uso.

\subsection{Descri��o dos Casos de Uso}

Nesta se��o s�o descritos os casos de uso levantados a partir dos requisitos funcionais.

\newpage

\subsubsection{Manter Usu�rios}

A tabela \ref{imgManterUsuarios} descreve o caso de uso de manter o cadastro, edi��o e remo��o das informa��es dos usu�rios, possui um �nico ator, o ``Administrador", a pr�-condi��o � ser um administrador do sistema.

\FloatBarrier
\begin{table}[!htb]
	\centering
	\caption{Caso de Uso Manter Usu�rios}
	\includegraphics[width=15cm]{UML/casos-de-uso/1.png}
	\label{imgManterUsuarios}
	Fonte: (AUTOR, 2015)
\end{table}
\FloatBarrier

\newpage

\subsubsection{Manter Alunos}

A tabela \ref{imgManterAlunos} descreve o caso de uso de manter o cadastro, edi��o e remo��o das informa��es dos alunos, possui um �nico ator, o ``Administrador", possui duas pr�-condi��es, ser administrador do sistema e o aluno precisa estar associado a um usu�rio.

\FloatBarrier
\begin{table}[!htb]
	\centering
	\caption{Caso de Uso Manter Alunos}
	\includegraphics[width=15cm]{UML/casos-de-uso/2.png}
	\label{imgManterAlunos}
	Fonte: (AUTOR, 2015)
\end{table}
\FloatBarrier

\newpage

\subsubsection{Manter Professores}

A tabela \ref{imgManterAlunos} descreve o caso de uso de manter o cadastro, edi��o e remo��o das informa��es dos professores, possui um �nico ator, o ``Administrador", possui duas pr�-condi��es, ser administrador do sistema e o professor precisa estar associado a um usu�rio.

\FloatBarrier
\begin{table}[!htb]
	\centering
	\caption{Caso de Uso Manter Professores}
	\includegraphics[width=15cm]{UML/casos-de-uso/3.png}
	\label{imgManterProfessores}
	Fonte: (AUTOR, 2015)
\end{table}
\FloatBarrier

\newpage

\subsubsection{Manter Administradores}

A tabela \ref{imgManterAdministradores} descreve o caso de uso de manter o cadastro, edi��o e remo��o das informa��es dos administradores do sistema, possui um �nico ator, o ``Administrador", possui duas pr�-condi��es, ser um administrador do sistema e possuir um usu�rio para adicionar como administrador do sistema.

\FloatBarrier
\begin{table}[!htb]
	\centering
	\caption{Caso de Uso Manter Administradores}
	\includegraphics[width=15cm]{UML/casos-de-uso/4.png}
	\label{imgManterAdministradores}
	Fonte: (AUTOR, 2015)
\end{table}
\FloatBarrier

\newpage

\subsubsection{Manter Tipo de Problema}

A tabela \ref{imgManterTipoProblemas} descreve o caso de uso de manter o cadastro, edi��o e remo��o das informa��es de um tipo de problema, um tipo de problema serve para agrupar problemas que possuem rela��o.

\FloatBarrier
\begin{table}[!htb]
	\centering
	\caption{Caso de Uso Manter Tipo de Problema}
	\includegraphics[width=15cm]{UML/casos-de-uso/5.png}
	\label{imgManterTipoProblemas}
	Fonte: (AUTOR, 2015)
\end{table}
\FloatBarrier

\newpage

\subsubsection{Manter Problemas}

A tabela \ref{imgManterProblemas} descreve o caso de uso de manter o cadastro, edi��o e remo��o das informa��es de problemas.

\FloatBarrier
\begin{table}[!htb]
	\centering
	\caption{Caso de Uso Manter Problemas}
	\includegraphics[width=15cm]{UML/casos-de-uso/6.png}
	\label{imgManterProblemas}
	Fonte: (AUTOR, 2015)
\end{table}
\FloatBarrier

\newpage

\subsubsection{Manter Palavras-chave}

A tabela \ref{imgManterEntradasSaidas} descreve o caso de uso de manter o cadastro, edi��o e remo��o das informa��es das palavras-chave de um ou mais problemas.

\FloatBarrier
\begin{table}[!htb]
	\centering
	\caption{Caso de Uso Manter Palavras-chave}
	\includegraphics[width=15cm]{UML/casos-de-uso/7.png}
	\label{imgManterPalavrasChaves}
	Fonte: (AUTOR, 2015)
\end{table}
\FloatBarrier

\newpage

\subsubsection{Manter Entradas e Sa�das}

A tabela \ref{imgManterEntradasSaidas} descreve o caso de uso de manter o cadastro, edi��o e remo��o das informa��es das entradas e sa�das de um determinado problema.

\FloatBarrier
\begin{table}[!htb]
	\centering
	\caption{Caso de Uso Manter Entrada e Sa�das}
	\includegraphics[width=15cm]{UML/casos-de-uso/8.png}
	\label{imgManterEntradasSaidas}
	Fonte: (AUTOR, 2015)
\end{table}
\FloatBarrier

\newpage

\subsubsection{Manter Solu��es de Problemas}

A tabela \ref{imgManterSolucoesProblemas} descreve o caso de uso de manter o cadastro, edi��o e remo��o das solu��es de um determinado problema. As solu��es podem ser cadastradas por administradores, professores e/ou alunos.

\FloatBarrier
\begin{table}[!htb]
	\centering
	\caption{Caso de Uso Manter Solu��es de Problemas}
	\includegraphics[width=15cm]{UML/casos-de-uso/9.png}
	\label{imgManterSolucoesProblemas}
	Fonte: (AUTOR, 2015)
\end{table}
\FloatBarrier

\newpage

\subsubsection{Manter Listas de Problemas}

A tabela \ref{imgManterListaProblemas} descreve o caso de uso de manter o cadastro, edi��o e remo��o das listas de problemas, elas agrupam problemas para serem aplicados a um grupo e/ou aluno.

\FloatBarrier
\begin{table}[!htb]
	\centering
	\caption{Caso de Uso Manter Listas de Problemas}
	\includegraphics[width=15cm]{UML/casos-de-uso/10.png}
	\label{imgManterListaProblemas}
	Fonte: (AUTOR, 2015)
\end{table}
\FloatBarrier

\newpage

\subsubsection{Manter Institui��es}

A tabela \ref{imgManterInstituicoes} descreve o caso de uso de manter o cadastro, edi��o e remo��o das institui��es que podem ser utilizadas no cadastro do professor.

\FloatBarrier
\begin{table}[!htb]
	\centering
	\caption{Caso de Uso Manter Institui��es}
	\includegraphics[width=15cm]{UML/casos-de-uso/11.png}
	\label{imgManterInstituicoes}
	Fonte: (AUTOR, 2015)
\end{table}
\FloatBarrier

\newpage

\subsubsection{Manter Grupos}

A tabela \ref{imgManterGrupos} descreve o caso de uso de manter o cadastro, edi��o e remo��o de grupos, eles servem para agrupar alunos e professores, onde o professores e administradores podem criar listas de problemas espec�ficos para os grupos ou alunos.

\FloatBarrier
\begin{table}[!htb]
	\centering
	\caption{Caso de Uso Manter Grupos}
	\includegraphics[width=15cm]{UML/casos-de-uso/12.png}
	\label{imgManterGrupos}
	Fonte: (AUTOR, 2015)
\end{table}
\FloatBarrier

\newpage

\subsubsection{Manter Permiss�es}

A tabela \ref{imgManterPermissoes} descreve o caso de uso de manter o cadastro, edi��o e remo��o das permiss�es do portal de algoritmos, essas permiss�es servem para permitir acesso a determinados cadastros do sistema.

\FloatBarrier
\begin{table}[!htb]
	\centering
	\caption{Caso de Uso Manter Permiss�es}
	\includegraphics[width=15cm]{UML/casos-de-uso/13.png}
	\label{imgManterPermissoes}
	Fonte: (AUTOR, 2015)
\end{table}
\FloatBarrier

\newpage

\subsubsection{Manter Grupos de Administradores}

A tabela \ref{imgManterGruposAdministradores} descreve o caso de uso de manter o cadastro, edi��o e remo��o de grupos de administradores. � um agrupador de usu�rios que tem a permiss�o de acessar o gerenciamento do portal de algoritmos.

\FloatBarrier
\begin{table}[!htb]
	\centering
	\caption{Caso de Uso Manter Grupos de Administradores}
	\includegraphics[width=15cm]{UML/casos-de-uso/14.png}
	\label{imgManterGruposAdministradores}
	Fonte: (AUTOR, 2015)
\end{table}
\FloatBarrier

\newpage

\subsubsection{Manter Pa�ses}

A tabela \ref{imgManterPaises} descreve o caso de uso de manter  o cadastro, edi��o e remo��o de pa�ses.

\FloatBarrier
\begin{table}[!htb]
	\centering
	\caption{Caso de Uso Manter Pa�ses}
	\includegraphics[width=15cm]{UML/casos-de-uso/15.png}
	\label{imgManterPaises}
	Fonte: (AUTOR, 2015)
\end{table}
\FloatBarrier

\newpage

\subsubsection{Manter Estados}

A tabela \ref{imgManterEstados} descreve o caso de uso de manter  o cadastro, edi��o e remo��o de estados.

\FloatBarrier
\begin{table}[!htb]
	\centering
	\caption{Caso de Uso Manter Estados}
	\includegraphics[width=15cm]{UML/casos-de-uso/16.png}
	\label{imgManterEstados}
	Fonte: (AUTOR, 2015)
\end{table}
\FloatBarrier

\newpage

\subsubsection{Manter Cidades}

A tabela \ref{imgManterCidades} descreve o caso de uso de manter  o cadastro, edi��o e remo��o de cidades.

\FloatBarrier
\begin{table}[!htb]
	\centering
	\caption{Caso de Uso Manter Cidades}
	\includegraphics[width=15cm]{UML/casos-de-uso/17.png}
	\label{imgManterCidades}
	Fonte: (AUTOR, 2015)
\end{table}
\FloatBarrier

\newpage

\subsubsection{Chat Online}

A tabela \ref{imgManterChat} descreve o caso de uso do chat online, ele servir� para comunica��o interna em tempo real dentro do portal de algoritmos.

\FloatBarrier
\begin{table}[!htb]
	\centering
	\caption{Caso de Uso Chat Online}
	\includegraphics[width=15cm]{UML/casos-de-uso/18.png}
	\label{imgManterChat}
	Fonte: (AUTOR, 2015)
\end{table}
\FloatBarrier

\newpage

\subsubsection{Manter suas Informa��es Pessoais}

A tabela \ref{imgManterPerfil} descreve o caso de uso de cadastro, edi��o e remo��o das informa��es pessoais do aluno.

\FloatBarrier
\begin{table}[!htb]
	\centering
	\caption{Caso de Uso Aluno Manter suas Informa��es Pessoais}
	\includegraphics[width=15cm]{UML/casos-de-uso/19.png}
	\label{imgManterPerfil}
	Fonte: (AUTOR, 2015)
\end{table}
\FloatBarrier

Foram apresentados os casos de uso que ser�o utilizados para a evolu��o do portal de algoritmos.

A seguir s�o apresentados os prot�tipos de interface do portal de algoritmos.

\newpage

\section{Prot�tipos de Interfaces Gr�ficas}\label{scPrototipos}

Os prot�tipos exibidos nessa se��o procuram seguir os seguintes princ�pios de usabilidade conforme \citet{Benyon2011}: 

\begin{enumerate}
	\item Visibilidade
		\subitem Garante que as coisas sejam vis�veis, de forma que as pessoas possam ver quais fun��es est�o dispon�veis e o que o sistema est� fazendo atualmente.
	\item Consist�ncia
		\subitem Mant�m consit�ncia no uso de caracteristicas de \textit{design} e com sistema semelhantes e m�todo-padr�o de trabalho.
	\item Familiaridade
		\subitem Utilizar linguagem e s�mbolos com os quais os futuros usu�rios est�o familiarizados.
	\item \textit{Affordance}
		\subitem Criar \textit{design} de forma que fique claro para o qu� elas servem.
	\item Navega��o
		\subitem Proporcione suporte para que as pessoas possam se movimentar pelo sistema.
	\item Retorno
		\subitem Retorne informa��o rapidamente a informa��o do sistema para as pessoas, para que elas saibam que efeito suas a��es causaram.
	\item Estilo
		\subitem Designs devem ser elegantes e atraentes.
\end{enumerate}

\newpage

\subsection{Cadastro de Aluno}

A figura \ref{mkCadastroAluno} � o prot�tipo de interface gr�fica de cadastro de um novo aluno.

\FloatBarrier
\begin{figure}[!htb]
	\centering
	\caption{Prot�tipo de Interface Gr�fica de Cadastro de Aluno}
	\includegraphics[width=15cm]{mockups/1.png}
	\label{mkCadastroAluno}
	Fonte: (AUTOR, 2015)
\end{figure}
\FloatBarrier

Nessa interface gr�fica o aluno preenche todos os campos abaixo:

\begin{itemize}
	\item Nome.
	\item Sobrenome.
	\item E-mail.
	\item Login.
		\subitem O login � seu \textit{username} que ser� utilizado para acessar o portal de algoritmos.
	\item Senha.
	\item Repita a senha.
		\subitem Nesse campo faz-se a verifica��o se a senha digitada anteriormente corresponde a essa.
	\item Quem � voc�?
		\subitem Escolhe uma das op��es dispon�veis, como por exemplo: Aluno da \ac{UCS}.
	\item Pa�s.
	\item Estado.
		\subitem Somente exibir� as op��es depois que selecionar um Pa�s.
	\item Cidade
		\subitem Somente exibir� as op��es depois que selecionar um Estado.
\end{itemize}

Ap�s o preenchimento de todos os campos, o alunos clica em ``Registrar" \ e seu cadastro estar� realizado.

\subsection{Autentica��o}

A figura \ref{mkAutenticacao} � o prot�tipo de interface gr�fica de autentica��o.

\FloatBarrier
\begin{figure}[!htb]
	\centering
	\caption{Prot�tipo de Interface Gr�fica de Autentica��o}
	\includegraphics[width=15cm]{mockups/5.png}
	\label{mkAutenticacao}
	Fonte: (AUTOR, 2015)
\end{figure}
\FloatBarrier

Nessa interface gr�fica o aluno, professor ou administrador realiza a autentica��o no portal de algoritmos. A autentica��o somente � autorizada para usu�rios ainda ativos no sistema.

O seu funcionamento � simples, o usu�rio preenche os campos de ``Usu�rio ou E-mail" \  e ``Senha" \  com as seguintes informa��es, e-mail ou usu�rio e senha, o usu�rio estando ativo no sistema ele ser� redirecionado para a p�gina inicial do sistema.

\newpage 

\subsection{Recupera��o de Senha}

A figura \ref{mkRecuperacaoSenha} � o prot�tipo de interface gr�fica de recupera��o de senha.

\FloatBarrier
\begin{figure}[!htb]
	\centering
	\caption{Prot�tipo de Interface Gr�fica de Recupera��o de Senha}
	\includegraphics[width=15cm]{mockups/4.png}
	\label{mkRecuperacaoSenha}
	Fonte: (AUTOR, 2015)
\end{figure}
\FloatBarrier

Nessa interface o usu�rio � capaz de solicitar uma nova senha somente preenchendo seu e-mail cadastrado no portal de algoritmos.

Ap�s o preenchimento do e-mail o usu�rio clica em ``Enviar", fazendo com que o sistema dispare um e-mail com a senha nova e anulando a senha antiga.

\newpage

\subsection{Boas Vindas do Aluno}

A figura \ref{mkBoasVindasAlunos} � o prot�tipo de interface gr�fica de Boas Vindas do Aluno.

\FloatBarrier
\begin{figure}[!htb]
	\centering
	\caption{Prot�tipo de Interface Gr�fica de Boas Vindas do Aluno}
	\includegraphics[width=15cm]{mockups/2.png}
	\label{mkBoasVindasAlunos}
	Fonte: (AUTOR, 2015)
\end{figure}
\FloatBarrier

Essa interface gr�fica � exibida ap�s a autentica��o com sucesso do Aluno. � exibido um v�deo explicativo de utiliza��o do novo portal de algoritmos.

\newpage

\subsection{Boas Vindas ao Administrador e Professor}

A figura \ref{mkBoasVindasAdministrador} � o prot�tipo de interface gr�fica de Boas Vindas do Administrador e Professor.

\FloatBarrier
\begin{figure}[!htb]
	\centering
	\caption{Prot�tipo de Interface Gr�fica de Boas Vindas do Administrador e Professor}
	\includegraphics[width=15cm]{mockups/3.png}
	\label{mkBoasVindasAdministrador}
	Fonte: (AUTOR, 2015)
\end{figure}
\FloatBarrier

Essa interface gr�fica � exibida ap�s a autentica��o com sucesso do Administrador ou Professor. Nessa interface � exibida uma �rvore com links para as demais interfaces gr�ficas, agrupadas por contextos.

\newpage

\subsection{Problemas}\label{scListaProblemas}

A figura \ref{mkListaProblemas} � o prot�tipo de interface gr�fica da listagem de problemas. Os problemas s�o agrupados por tipos de problemas.

\FloatBarrier
\begin{figure}[!htb]
	\centering
	\caption{Prot�tipo de Interface Gr�fica de Problemas}
	\includegraphics[width=15cm]{mockups/6.png}
	\label{mkListaProblemas}
	Fonte: (AUTOR, 2015)
\end{figure}
\FloatBarrier

Essa interface gr�fica possui as seguintes a��es:

\begin{itemize}
	\item Pesquisa livre.
		\subitem O administrador pode realizar a filtragem por qualquer palavra.
	\item Novo Tipo de Problema.
		\subitem Selecionando essa a��o, o administrador � direcionado para a interface gr�fica da figura \ref{mkNovoTipoProblema}.
	\item Novo Problema.
		\subitem Selecionando essa a��o, o adminsitrador � direcionado para a interface gr�fica da figura \ref{mkNovoProblema}.
	\item Editar Problema.
		\subitem Selecionando essa a��o, o administrador � direcionado para a interface gr�fica da figura \ref{mkNovoProblema}.
	\item Remover Problema.
		\subitem Selecionando essa a��o, o administrador remove o problema.
	\item Visualizar Solu��o.
		\subitem Selecionando essa a��o, o administrador � direcionado  para a interface gr�fica da figura \ref{mkVisualizacaoSolucao1}.
\end{itemize}

\subsection{Tipo de Problema}

A figura \ref{mkNovoTipoProblema} � o prot�tipo de interface gr�fica de cadastro de novo tipo de problema.

\FloatBarrier
\begin{figure}[!htb]
	\centering
	\caption{Prot�tipo de Interface Gr�fica de Novo Tipo de Problema}
	\includegraphics[width=15cm]{mockups/7.png}
	\label{mkNovoTipoProblema}
	Fonte: (AUTOR, 2015)
\end{figure}
\FloatBarrier

Nessa interface gr�fica o administrador preenche somente um ``Nome" \  para o tipo de problema. Ap�s o preenchimento o administrador clica em ``Salvar" \  e � redirecionado para listagem de problemas que foi descrito na se��o \ref{scListaProblemas}.

\newpage

\subsection{Cadastro e Edi��o de Problema}

A figura \ref{mkNovoProblema} � o prot�tipo de interface gr�fica de cadastro e/ou edi��o de um novo problema.

\FloatBarrier
\begin{figure}[!htb]
	\centering
	\caption{Prot�tipo de Interface Gr�fica de Novo Problema}
	\includegraphics[width=15cm]{mockups/8.png}
	\label{mkNovoProblema}
	Fonte: (AUTOR, 2015)
\end{figure}
\FloatBarrier

A interface est� dividida em tr�s abas, Problema, Palavras Chave e Dados de Testes, cada uma dessas abas est�o descritas abaixo:

\begin{itemize}
	\item Problema.
		\subitem O cadastro ou edi��o do problema consiste em preencher os campos, ``Tipo de Problema", ``Nome do Problema", ``Dica" \ e ``N�vel", o ``Custo" \ � correspondente ao menor custo de alguma solu��o daquele problema.
	\item Palavras Chaves.
		\subitem O cadastro de palavras-chave � descrito na se��o \ref{scPalavraChave}
	\item Dados de Testes.
		\subitem O cadastro de dados de testes � descrito na se��o \ref{scDadosTestes}
\end{itemize}

Toda a a��o nessa interface deve ser gravada ao clicar no bot�o ``Salvar".

\subsection{Palavras-chave}\label{scPalavraChave}

A figura \ref{mkPalavrasChaves} � o prot�tipo de interface gr�fica de cadastro e/ou edi��o de palavras-chave de um problema.

\FloatBarrier
\begin{figure}[!htb]
	\centering
	\caption{Prot�tipo de Interface Gr�fica de Palavras Chaves}
	\includegraphics[width=15cm]{mockups/9.png}
	\label{mkPalavrasChaves}
	Fonte: (AUTOR, 2015)
\end{figure}
\FloatBarrier

Nessa interface o administrador preenche o campo ``Palavra-chave" \  e clica em ``Adicionar". Fazendo essa a��o ele associa a nova palavra-chave ao problema que ele est� cadastrando ou editando. O administrador pode tamb�m editar ou remover uma palavra-chave j� cadastrada, para isso � preciso selecionar um das a��es, ``Editar" \ ou ``Remover".

\newpage

\subsection{Dados de Testes}\label{scDadosTestes}

A figura \ref{mkDadosTestes} � o prot�tipo de interface gr�fica de cadastro e/ou edi��o de dados de testes de um problema.

\FloatBarrier
\begin{figure}[!htb]
	\centering
	\caption{Prot�tipo de Interface Gr�fica de Dados de Testes}
	\includegraphics[width=15cm]{mockups/10.png}
	\label{mkDadosTestes}
	Fonte: (AUTOR, 2015)
\end{figure}
\FloatBarrier

Nessa interface o administrador preenche os campos ``Entrada" \ e ``Sa�da" \  e clica em ``Adicionar", fazendo essa a��o ele associa o novo dado de teste ao problema que ele est� cadastrando ou editando. O administrador pode tamb�m editar ou remover um dado de teste j� cadastrado, para isso � preciso selecionar um das a��es, ``Editar" \  ou ``Remover". Um problema pode ter apenas dados de testes de ``Entrada" \ ou somente de ``Sa�da".

\newpage

\subsection{Visualiza��o de Solu��o partindo da listagem de Problemas}

A figura \ref{mkVisualizacaoSolucao1} � o prot�tipo de interface gr�fica de visualiza��o de uma solu��o de um problema. Podemos observar no topo da interface que � exibido o caminho que foi utilizado para chegar a essa solu��o. Em outras interfaces que ser�o descritas no decorrer deste trabalho, esse caminho pode variar de acordo com o contexto do gerenciamento do portal de algoritmos.

\FloatBarrier
\begin{figure}[!htb]
	\centering
	\caption{Prot�tipo de Interface Gr�fica de Visualiza��o de Solu��o do Problema}
	\includegraphics[width=15cm]{mockups/11.png}
	\label{mkVisualizacaoSolucao1}
	Fonte: (AUTOR, 2015)
\end{figure}
\FloatBarrier

\newpage

\subsection{Lista de Grupos}

A figura \ref{mkListaGrupos} � o prot�tipo de interface gr�fica de listagem dos grupos ja cadastrados. Essa listagem corresponde a todos os grupos, caso seja um administrador com total acesso, ou apenas aos grupos cadastrados por um professor, sendo que o professor somente ter� na listagem os grupos que ele cadastrou.

\FloatBarrier
\begin{figure}[!htb]
	\centering
	\caption{Prot�tipo de Interface Gr�fica de Grupos}
	\includegraphics[width=15cm]{mockups/12.png}
	\label{mkListaGrupos}
	Fonte: (AUTOR, 2015)
\end{figure}
\FloatBarrier

Nessa interface gr�fica possu�mos as seguintes a��es:

\begin{itemize}
	\item Pesquisa Livre.
		\subitem O administrador pode realizar a pesquisa por qualquer palavra.
	\item Novo Grupo.
		\subitem Selecionando essa a��o, o administrador � direcionado para a interface gr�fica da figura \ref{mkCadastroGrupo}
	\item Editar Grupo.
		\subitem Selecionando essa a��o, o administrador � direcionado para a interface gr�fica da figura \ref{mkCadastroGrupo}
	\item Remover Grupo.
		\subitem Selecionando essa a��o, o administrador remove somente o grupo, sendo assim, n�o remove problemas, solu��es e/ou usu�rios.
	\item Visualizar Solu��o.
		\subitem Selecionando essa a��o, o administrador � direcionado para a interface gr�fica da figura \ref{mkVisualizacaoSolucao2}
\end{itemize}

\subsection{Cadastro e Edi��o de Grupo}

A figura \ref{mkCadastroGrupo} � o prot�tipo de interface gr�fica de cadastro e edi��o de um grupo.

\FloatBarrier
\begin{figure}[!htb]
	\centering
	\caption{Prot�tipo de Interface Gr�fica de Cadastro de Grupo}
	\includegraphics[width=15cm]{mockups/13.png}
	\label{mkCadastroGrupo}
	Fonte: (AUTOR, 2015)
\end{figure}
\FloatBarrier

A interface gr�fica est� dividida em tr�s abas, Grupo, Participantes, Lista de Problemas, cada uma dessas abas est�o descritas abaixo:

\begin{itemize}
	\item Grupo.
		\subitem Para o cadastro de um novo grupo, basta preencher os campos, Nome e Descri��o.
	\item Participantes.
		\subitem O cadastro de participantes � descrito na figura \ref{mkParticipantes}
	\item Lista de Problemas.
		\subitem O cadastro de lista de problemas � descrito na figura \ref{mkGrupoListaProblemas}
\end{itemize}

\newpage

\subsection{Adicionar Participantes}

A figura \ref{mkParticipantes} � o prot�tipo de interface gr�fica de associa��o de participantes em um grupo.

\FloatBarrier
\begin{figure}[!htb]
	\centering
	\caption{Prot�tipo de Interface Gr�fica de Cadastro de Participantes no Grupo}
	\includegraphics[width=15cm]{mockups/14.png}
	\label{mkParticipantes}
	Fonte: (AUTOR, 2015)
\end{figure}
\FloatBarrier

Utilizando os bot�es centrais, � poss�vel adicionar ou remover participantes, pode-se adicionar ou remover todos ou um e/ou mais por vez.

\newpage

\subsection{Adicionar Lista de Problemas}

A figura \ref{mkGrupoListaProblemas} � o prot�tipo de interface gr�fica de associa��o de lista de problemas em um grupo.

\FloatBarrier
\begin{figure}[!htb]
	\centering
	\caption{Prot�tipo de Interface Gr�fica de Cadastro de Lista de Problemas no Grupo}
	\includegraphics[width=15cm]{mockups/15.png}
	\label{mkGrupoListaProblemas}
	Fonte: (AUTOR, 2015)
\end{figure}
\FloatBarrier

Utilizando os bot�es centrais, � poss�vel adicionar ou remover listas de problemas, pode-se adicionar ou remover todos ou um e/ou mais por vez.

\newpage

\subsection{Visualiza��o de Solu��o partindo da listagem de Grupos}

A figura \ref{mkVisualizacaoSolucao2} � o prot�tipo de interface gr�fica de visualiza��o de uma solu��o de um problema. Podemos observar no topo da interface que � exibido o caminho que foi utilizado para chegar a essa solu��o

\FloatBarrier
\begin{figure}[!htb]
	\centering
	\caption{Prot�tipo de Interface Gr�fica de Visualiza��o de Solu��o dos Grupos}
	\includegraphics[width=15cm]{mockups/16.png}
	\label{mkVisualizacaoSolucao2}
	Fonte: (AUTOR, 2015)
\end{figure}
\FloatBarrier

\newpage

\subsection{Lista de Alunos}

A figura \ref{mkListaAlunos} � o prot�tipo de interface gr�fica da listagem de alunos.

\FloatBarrier
\begin{figure}[!htb]
	\centering
	\caption{Prot�tipo de Interface Gr�fica de Listagem de Alunos}
	\includegraphics[width=15cm]{mockups/17.png}
	\label{mkListaAlunos}
	Fonte: (AUTOR, 2015)
\end{figure}
\FloatBarrier

Nessa interface gr�fica encontramos as seguintes a��es:

\begin{itemize}
	\item Pesquisa Livre.
		\subitem O administrador pode realizar a pesquisa por qualquer palavra.
	\item Novo Aluno.
		\subitem Selecionando essa a��o, o administrador � direcionado para a interface gr�fica da figura \ref{mkCadastroAlunoAdmin}
	\item Editar Aluno.
		\subitem Selecionando essa a��o, o administrador � direcionado para a interface gr�fica da figura \ref{mkCadastroAlunoAdmin}
	\item Remover Aluno.
		\subitem Selecionando essa a��o, o administrador remove o aluno.
	\item Visualizar Solu��o.
		\subitem Selecionando essa a��o, o administrador � direcionado  para a interface gr�fica da figura \ref{mkVisualizacaoSolucao3}.
\end{itemize}

\newpage

\subsection{Cadastro de Alunos}

A figura \ref{mkCadastroAlunoAdmin} � o prot�tipo de interface gr�fica do cadastro e edi��o de alunos.

\FloatBarrier
\begin{figure}[!htb]
	\centering
	\caption{Prot�tipo de Interface Gr�fica de Cadastro de Alunos}
	\includegraphics[width=15cm]{mockups/38.png}
	\label{mkCadastroAlunoAdmin}
	Fonte: (AUTOR, 2015)
\end{figure}
\FloatBarrier

A �nica informa��o que � necess�rio preencher � selecionar um Usu�rio, esse usu�rio ficar� associado ao cadastro do aluno.

\newpage

\subsection{Visualiza��o de Solu��o partindo da listagem de Alunos}

A figura \ref{mkVisualizacaoSolucao3} � o prot�tipo de interface gr�fica de visualiza��o de uma solu��o de um problema. Podemos observar no topo da interface que � exibido  o caminho que foi utilizado para chegar a essa solu��o.

\FloatBarrier
\begin{figure}[!htb]
	\centering
	\caption{Prot�tipo de Interface Gr�fica de Visualiza��o de Solu��o de Alunos}
	\includegraphics[width=15cm]{mockups/18.png}
	\label{mkVisualizacaoSolucao3}
	Fonte: (AUTOR, 2015)
\end{figure}
\FloatBarrier

\newpage

\subsection{Listas de Problemas}

A figura \ref{mkListaProblemasListagem} � o prot�tipo de interface da listagem de listas de problemas. Essa interface lista todas as listas de problemas j� cadastradas no portal de algoritmos.

\FloatBarrier
\begin{figure}[!htb]
	\centering
	\caption{Prot�tipo de Interface Gr�fica de Lista de Problemas}
	\includegraphics[width=15cm]{mockups/23.png}
	\label{mkListaProblemasListagem}
	Fonte: (AUTOR, 2015)
\end{figure}
\FloatBarrier

Nessa interface gr�fica temos as seguintes a��es:

\begin{itemize}
	\item Pesquisa Livre.
		\subitem O administrador pode realizar a pesquisa por qualquer palavra.
	\item Nova Lista.
		\subitem Selecionado essa a��o, o administrador � direcionado para a interface gr�fica da figura \ref{mkCadastroListaProblemas}
	\item Editar Lista.
		\subitem Selecionado essa a��o, o administrador � direcionado para a interface gr�fica da figura \ref{mkCadastroListaProblemas}
	\item Remover Lista.
		\subitem Selecionando essa a��o, o administrador remove uma lista.
	\item Visualizar Solu��o.
		\subitem Selecionando essa a��o, o administrador � direcionado para a interface gr�fica da figura \ref{mkVisualizacaoSolucao4}
\end{itemize}

\newpage

\subsection{Cadastro e Edi��o de Lista de Problemas}

A figura \ref{mkCadastroListaProblemas} � o prot�tipo de interface de cadastro e edi��o de lista de problemas.

\FloatBarrier
\begin{figure}[!htb]
	\centering
	\caption{Prot�tipo de Interface Gr�fica de Cadastro e Edi��o de Lista de Problemas}
	\includegraphics[width=15cm]{mockups/25.png}
	\label{mkCadastroListaProblemas}
	Fonte: (AUTOR, 2015)
\end{figure}
\FloatBarrier

Para cadastrar ou editar, o administrador deve preencher um nome para lista e selecionar os problemas conforme as a��es dispon�veis na interface.

\newpage

\subsection{Visualiza��o de Solu��o partindo da listagem de Listas de Problemas}

A figura \ref{mkVisualizacaoSolucao4} � o prot�tipo de inteface gr�fica de uma solu��o de um problema. Podemos observar no topo da interface que � exibido  o caminho que foi utilizado para chegar a essa solu��o.

\FloatBarrier
\begin{figure}[!htb]
	\centering
	\caption{Prot�tipo de Interface Gr�fica de Visualiza��o de Solu��o de Lista de Problemas}
	\includegraphics[width=15cm]{mockups/24.png}
	\label{mkVisualizacaoSolucao4}
	Fonte: (AUTOR, 2015)
\end{figure}
\FloatBarrier

\newpage

\subsection{Lista de Usu�rios}

A figura \ref{mkListaUsuario} � o prot�tipo de interface gr�fica da listagem dos usu�rios. Essa interface lista todos os usu�rios cadastrados no portal de algoritmos, incluindo os inativos.

\FloatBarrier
\begin{figure}[!htb]
	\centering
	\caption{Prot�tipo de Interface Gr�fica de Lista de Usu�rios}
	\includegraphics[width=15cm]{mockups/36.png}
	\label{mkListaUsuario}
	Fonte: (AUTOR, 2015)
\end{figure}
\FloatBarrier

Nessa interface gr�fica temos as seguintes a��es:

\begin{itemize}
	\item Pesquisa Livre.
		\subitem O administrador pode realizar a pesquisa por qualquer palavra.
	\item Novo Usu�rio.
		\subitem Selecionando essa a��o, o administrador � direcionado para a interface gr�fica da figura \ref{mkCadastroUsuario}.
	\item Editar Usu�rio.
		\subitem Selecionando essa a��o, o administrador � direcionado para a interface gr�fica da figura \ref{mkCadastroUsuario}.
	\item Remover Usu�rio.
		\subitem Selecionando essa a��o, o administrador remove um usu�rio.
\end{itemize}

\newpage

\subsection{Cadastro e Edi��o de Usu�rios}

A figura \ref{mkCadastroUsuario} � o prot�tipo de interface gr�fica do cadastro e edi��o dos usu�rios.

\FloatBarrier
\begin{figure}[!htb]
	\centering
	\caption{Prot�tipo de Interface Gr�fica de Cadastro e Edi��o de Usu�rios}
	\includegraphics[width=15cm]{mockups/37.png}
	\label{mkCadastroUsuario}
	Fonte: (AUTOR, 2015)
\end{figure}
\FloatBarrier

O administrador preenche todos os campos do cadastro e clica em ``Salvar", ap�s o administrador cadastrar ou editar o aluno o sistema retorna para listagem de usu�rios.

\newpage

\subsection{Lista de Administradores}

A figura \ref{mkListaAdministradores} � o prot�tipo de interface gr�fica da listagem dos administradores.

\FloatBarrier
\begin{figure}[!htb]
	\centering
	\caption{Prot�tipo de Interface Gr�fica de Lista de Administradores}
	\includegraphics[width=15cm]{mockups/26.png}
	\label{mkListaAdministradores}
	Fonte: (AUTOR, 2015)
\end{figure}
\FloatBarrier

Nessa interface gr�fica temos as seguintes a��es:

\begin{itemize}
	\item Pesquisa Livre.
		\subitem O administrador pode realizar a pesquisa por qualquer palavra.
	\item Novo Administrador.
		\subitem Selecionando essa a��o, o administrador � direcionado para a interface gr�fica da figura \ref{mkCadastroAdministradores}.
	\item Editar Administrador.
		\subitem Selecionando essa a��o, o administrador � direcionado para a interface gr�fica da figura \ref{mkCadastroAdministradores}.
	\item Remover Administrador.
		\subitem Selecionando essa a��o, o administrador remove um Administrador.
\end{itemize}

\newpage

\subsection{Cadastro e Edi��o de Administradores}

A figura \ref{mkCadastroAdministradores} � o prot�tipo de interface gr�fica do cadastro e edi��o dos administradores.

\FloatBarrier
\begin{figure}[!htb]
	\centering
	\caption{Prot�tipo de Interface Gr�fica de Cadastro e Edi��o de Administradores}
	\includegraphics[width=15cm]{mockups/27.png}
	\label{mkCadastroAdministradores}
	Fonte: (AUTOR, 2015)
\end{figure}
\FloatBarrier

O administrador seleciona um usu�rio para cadastrar como administrador, podendo tamb�m adicionar grupos de administradores, sendo que esses grupos possuem permiss�es diferentes para acessar o gerenciamento do portal de algoritmos. Ap�s preencher todo o cadastro o administrador clica em ``Salvar" \ e o sistema direciona para listagem de administradores.

\newpage

\subsection{Lista de Professores}

A figura \ref{mkListaProfessores} � o prot�tipo de interface gr�fica da listagem dos professores.

\FloatBarrier
\begin{figure}[!htb]
	\centering
	\caption{Prot�tipo de Interface Gr�fica de Lista de Professores}
	\includegraphics[width=15cm]{mockups/28.png}
	\label{mkListaProfessores}
	Fonte: (AUTOR, 2015)
\end{figure}
\FloatBarrier

Nessa interface gr�fica temos as seguintes a��es:

\begin{itemize}
	\item Pesquisa Livre.
		\subitem O administrador pode realizar a pesquisa por qualquer palavra.
	\item Novo Professor.
		\subitem Selecionando essa a��o, o administrador � direcionado para a interface gr�fica da figura \ref{mkCadastroProfessores}.
	\item Editar Professor.
		\subitem Selecionando essa a��o, o administrador � direcionado para a interface gr�fica da figura \ref{mkCadastroProfessores}.
	\item Remover Professor.
		\subitem Selecionando essa a��o, o administrador remove um Professor.
\end{itemize}

\newpage

\subsection{Cadastro e Edi��o de Professores}

A figura \ref{mkCadastroProfessores} � o prot�tipo de interface gr�fica do cadastro e edi��o dos professores.

\FloatBarrier
\begin{figure}[!htb]
	\centering
	\caption{Prot�tipo de Interface Gr�fica de Cadastro e Edi��o de Professores}
	\includegraphics[width=15cm]{mockups/29.png}
	\label{mkCadastroProfessores}
	Fonte: (AUTOR, 2015)
\end{figure}
\FloatBarrier

O administrador seleciona um usu�rio para cadastrar como professor, podendo tamb�m associar uma institui��o que ele pertence. Ap�s preencher todo o cadastro, o administrador clica em ``Salvar" \ e o sistema direciona para listagem de professores.

\newpage

\subsection{Lista de Institui��es}

A figura \ref{mkListaInstituicao} � o prot�tipo de interface gr�fica da listagem das Institui��es.

\FloatBarrier
\begin{figure}[!htb]
	\centering
	\caption{Prot�tipo de Interface Gr�fica de Lista de Institui��es}
	\includegraphics[width=15cm]{mockups/30.png}
	\label{mkListaInstituicao}
	Fonte: (AUTOR, 2015)
\end{figure}
\FloatBarrier

Nessa interface gr�fica temos as seguintes a��es:

\begin{itemize}
	\item Pesquisa Livre.
		\subitem O administrador pode realizar a pesquisa por qualquer palavra.
	\item Novo Institui��o.
		\subitem Selecionando essa a��o, o administrador � direcionado para a interface gr�fica da figura \ref{mkCadastroInstituicao}.
	\item Editar Institui��o.
		\subitem Selecionando essa a��o, o administrador � direcionado para a interface gr�fica da figura \ref{mkCadastroInstituicao}.
	\item Remover Institui��o.
		\subitem Selecionando essa a��o, o administrador remove uma Institui��o.
\end{itemize}

\newpage

\subsection{Cadastro e Edi��o de Institui��es}

A figura \ref{mkCadastroInstituicao} � o prot�tipo de interface gr�fica do cadastro e edi��o de institui��es.

\FloatBarrier
\begin{figure}[!htb]
	\centering
	\caption{Prot�tipo de Interface Gr�fica de Cadastro e Edi��o de Institui��es}
	\includegraphics[width=15cm]{mockups/31.png}
	\label{mkCadastroInstituicao}
	Fonte: (AUTOR, 2015)
\end{figure}
\FloatBarrier

O administrador preenche um nome e seleciona uma cidade para cadastrar ou editar uma institui��o. Ap�s preencher as informa��es, o administrador clica em Salvar" \ e o sistema direciona para listagem das institui��es.

\newpage

\subsection{Lista de Permiss�es}

A figura \ref{mkListaInstituicao} � o prot�tipo de interface gr�fica da listagem das Permiss�es.

\FloatBarrier
\begin{figure}[!htb]
	\centering
	\caption{Prot�tipo de Interface Gr�fica de Lista de Permiss�es}
	\includegraphics[width=15cm]{mockups/32.png}
	\label{mkListaPermissao}
	Fonte: (AUTOR, 2015)
\end{figure}
\FloatBarrier

Nessa interface gr�fica temos as seguintes a��es:

\begin{itemize}
	\item Pesquisa Livre.
		\subitem O administrador pode realizar a pesquisa por qualquer palavra.
	\item Nova Permiss�es.
		\subitem Selecionando essa a��o, o administrador � direcionado para a interface gr�fica da figura \ref{mkCadastroPermissao}.
	\item Editar Permiss�es.
		\subitem Selecionando essa a��o, o administrador � direcionado para a interface gr�fica da figura \ref{mkCadastroPermissao}.
	\item Remover Permiss�es.
		\subitem Selecionando essa a��o, o administrador remove uma Permiss�o.
\end{itemize}

\newpage

\subsection{Cadastro e Edi��o de Permiss�es}

A figura \ref{mkCadastroPermissao} � o prot�tipo de interface gr�fica do cadastro e edi��o de permiss�es.

\FloatBarrier
\begin{figure}[!htb]
	\centering
	\caption{Prot�tipo de Interface Gr�fica de Cadastro e Edi��o de Permiss�es}
	\includegraphics[width=15cm]{mockups/33.png}
	\label{mkCadastroPermissao}
	Fonte: (AUTOR, 2015)
\end{figure}
\FloatBarrier

O administrador preenche um nome e uma descri��o para cadastrar uma permiss�o, ap�s o preenchimento o administrador clica em ``Salvar" \ e o sistema direciona para a listagem das permiss�es

\newpage

\subsection{Lista de Grupos de Administradores}

A figura \ref{mkListaGrupoAdministradores} � o prot�tipo de interface gr�fica da listagem das Grupos de Administradores.

\FloatBarrier
\begin{figure}[!htb]
	\centering
	\caption{Prot�tipo de Interface Gr�fica de Lista de Grupos de Administradores}
	\includegraphics[width=15cm]{mockups/34.png}
	\label{mkListaGrupoAdministradores}
	Fonte: (AUTOR, 2015)
\end{figure}
\FloatBarrier

Nessa interface gr�fica temos as seguintes a��es:

\begin{itemize}
	\item Pesquisa Livre.
		\subitem O administrador pode realizar a pesquisa por qualquer palavra.
	\item Novo Grupo.
		\subitem Selecionando essa a��o, o administrador � direcionado para a interface gr�fica da figura \ref{mkCadastroGrupoAdministradores}.
	\item Editar Grupo.
		\subitem Selecionando essa a��o, o administrador � direcionado para a interface gr�fica da figura \ref{mkCadastroGrupoAdministradores}.
	\item Remover Grupo.
		\subitem Selecionando essa a��o, o administrador remove um Grupo de Administrador.
\end{itemize}

\newpage

\subsection{Cadastro e Edi��o de Grupos de Administradores}

A figura \ref{mkCadastroGrupoAdministradores} � o prot�tipo de interface gr�fica do cadastro e edi��o de permiss�es.

\FloatBarrier
\begin{figure}[!htb]
	\centering
	\caption{Prot�tipo de Interface Gr�fica de Cadastro e Edi��o de Grupos de Administradores}
	\includegraphics[width=15cm]{mockups/35.png}
	\label{mkCadastroGrupoAdministradores}
	Fonte: (AUTOR, 2015)
\end{figure}
\FloatBarrier

O administrador preenche um nome, uma descri��o e associa permiss�es a esse grupo. Ap�s preencher as informa��es o administrador clica em ``Salvar" \ e o sistema direciona para a listagem dos grupos de administradores.

\newpage

\subsection{Lista de Pa�ses, Estados e Cidades}

A figura \ref{mkListaPaises} � o prot�tipo de interface gr�fica da listagem de pa�ses, estados e cidades cadastradas no portal de algoritmos.

\FloatBarrier
\begin{figure}[!htb]
	\centering
	\caption{Prot�tipo de Interface Gr�fica de Listagem de Pa�ses, Estados e Cidades}
	\includegraphics[width=15cm]{mockups/19.png}
	\label{mkListaPaises}
	Fonte: (AUTOR, 2015)
\end{figure}
\FloatBarrier

Nessa interface encontramos as seguintes a��es:

\begin{itemize}
	\item Pesquisa Livre.
		\subitem O administrador pode realizar a pesquisa por qualquer palavra.
	\item Novo Pa�s.
		\subitem Selecionando essa a��o, o administrador � direcionado para a interface gr�fica da figura \ref{mkCadastroPais}.
	\item Editar Pa�s.
		\subitem Selecionando essa a��o, o administrador � direcionado para a interface gr�fica da figura \ref{mkCadastroPais}.
	\item Remover Pa�s.
		\subitem Selecionando essa a��o, o administrador remove um pa�s e todos os estados e cidades associados a ele.
	\item Novo Estado.
		\subitem Selecionando essa a��o, o administrador � direcionado para a interface gr�fica da figura \ref{mkCadastroEstado}.
	\item Editar Estado.
		\subitem Selecionando essa a��o, o administrador � direcionado para a interface gr�fica da figura \ref{mkCadastroEstado}.
	\item Remover Estado.
		\subitem Selecionando essa a��o, o administrador remove um estado e todas as cidades associadas a ele.
	\item Nova Cidade.
		\subitem Selecionando essa a��o, o administrador � direcionado para a interface gr�fica da figura \ref{mkCadastroCidade}.
	\item Editar Cidade.
		\subitem Selecionando essa a��o, o administrador � direcionado para a interface gr�fica da figura \ref{mkCadastroCidade}.
	\item Remover Cidade.
		\subitem Selecionado essa a��o, o administrador remove uma cidade.
\end{itemize}

\newpage

\subsection{Cadastro e Edi��o de Pa�s}

A figura \ref{mkCadastroPais} � o prot�tipo de interface gr�fica do cadastro e edi��o de um Pa�s.

\FloatBarrier
\begin{figure}[!htb]
	\centering
	\caption{Prot�tipo de Interface Gr�fica de Cadastro e Edi��o de Pa�s}
	\includegraphics[width=15cm]{mockups/20.png}
	\label{mkCadastroPais}
	Fonte: (AUTOR, 2015)
\end{figure}
\FloatBarrier

O administrador preenche o nome do Pa�s e sua Abrevia��o e clica em ``Salvar", ap�s retorna para listagem de pa�ses, estados e cidades.

\newpage

\subsection{Cadastro e Edi��o de Estado}

A figura \ref{mkCadastroEstado} � o prot�tipo de interface gr�fica do cadastro e edi��o de um Estado.

\FloatBarrier
\begin{figure}[!htb]
	\centering
	\caption{Prot�tipo de Interface Gr�fica de Cadastro e Edi��o de Estado}
	\includegraphics[width=15cm]{mockups/21.png}
	\label{mkCadastroEstado}
	Fonte: (AUTOR, 2015)
\end{figure}
\FloatBarrier

O administrador seleciona um Pa�s, preenche o nome do Estado e sua Abrevia��o e clica em ``Salvar", ap�s retorna para listagem de pa�ses, estados e cidades.

\newpage

\subsection{Cadastro e Edi��o de Cidade}

A figura \ref{mkCadastroEstado} � o prot�tipo de interface gr�fica do cadastro e edi��o de uma Cidade.

\FloatBarrier
\begin{figure}[!htb]
	\centering
	\caption{Prot�tipo de Interface Gr�fica de Cadastro e Edi��o de Cidade}
	\includegraphics[width=15cm]{mockups/22.png}
	\label{mkCadastroCidade}
	Fonte: (AUTOR, 2015)
\end{figure}
\FloatBarrier

O administrador seleciona um Pa�s, seleciona um Estado e preenche o nome da cidade e clica em ``Salvar", ap�s retorna para listagem de pa�ses, estados e cidades.

Nesta se��o foram apresentados todos os prot�tipos de interfaces gr�ficas, a seguir veremos os diagramas de robustez na qual � desmonstrada a liga��o entre as interfaces e as classes de dom�nio.

\section{Diagramas de Robustez}

Diagramas de Robustez tem por objetivo demonstrar a liga��o dos casos de uso com as interfaces gr�ficas e classes de dom�nio dentro do contexto da modelagem do \textit{software}.

\newpage

\subsection{Cadastro de Aluno}

A figura \ref{rbRegistroAluno} demonstra as associa��es das classes de dom�nio na interface gr�fica da figura \ref{mkCadastroAluno}.

\FloatBarrier
\begin{figure}[!htb]
	\centering
	\caption{Diagrama de Robustez de Cadastro de Aluno}
	\includegraphics[width=15cm]{UML/robustez/1.png}
	\label{rbRegistroAluno}
	Fonte: (AUTOR, 2015)
\end{figure}
\FloatBarrier

\subsection{Autentica��o}

A figura \ref{rbAutenticacao} demonstra as associa��es das classes de dom�nio na interface gr�fica da figura \ref{mkAutenticacao}.

\FloatBarrier
\begin{figure}[!htb]
	\centering
	\caption{Diagrama de Robustez de Autentica��o}
	\includegraphics[width=15cm]{UML/robustez/2.png}
	\label{rbAutenticacao}
	Fonte: (AUTOR, 2015)
\end{figure}
\FloatBarrier

\newpage

\subsection{Recupera��o de Senha}

A figura \ref{rbRecuperacaoSenha} demonstra as associa��es das classes de dom�nio na interface gr�fica da figura \ref{mkRecuperacaoSenha}.

\FloatBarrier
\begin{figure}[!htb]
	\centering
	\caption{Diagrama de Robustez de Recupera��o de Senha}
	\includegraphics[width=15cm]{UML/robustez/3.png}
	\label{rbRecuperacaoSenha}
	Fonte: (AUTOR, 2015)
\end{figure}
\FloatBarrier

\subsection{Boas Vindas do Aluno}

A figura \ref{rbBoasVindasAluno} demonstra as associa��es das classes de dom�nio na interface gr�fica da figura \ref{mkBoasVindasAlunos}.

\FloatBarrier
\begin{figure}[!htb]
	\centering
	\caption{Diagrama de Robustez de Boas Vindas do Aluno}
	\includegraphics[width=15cm]{UML/robustez/4.png}
	\label{rbBoasVindasAluno}
	Fonte: (AUTOR, 2015)
\end{figure}
\FloatBarrier

\newpage

\subsection{Boas Vindas do Administrador e do Professor}

A figura \ref{rbBoasVindasAdministrador} demonstra as associa��es das classes de dom�nio na interface gr�fica da figura \ref{mkBoasVindasAdministrador}.

\FloatBarrier
\begin{figure}[!htb]
	\centering
	\caption{Diagrama de Robustez de Boas Vindas Administrador}
	\includegraphics[width=15cm]{UML/robustez/5.png}
	\label{rbBoasVindasAdministrador}
	Fonte: (AUTOR, 2015)
\end{figure}
\FloatBarrier

\subsection{Problemas}

A figura \ref{rbProblemas} demonstra as associa��es das classes de dom�nio na interface gr�fica da figura \ref{mkListaProblemas}.

\FloatBarrier
\begin{figure}[!htb]
	\centering
	\caption{Diagrama de Robustez de Problemas}
	\includegraphics[width=15cm]{UML/robustez/6.png}
	\label{rbProblemas}
	Fonte: (AUTOR, 2015)
\end{figure}
\FloatBarrier

\subsection{Tipo de Problema}

A figura \ref{rbTipoProblema} demonstra as associa��es das classes de dom�nio na interface gr�fica da figura \ref{mkNovoTipoProblema}.

\FloatBarrier
\begin{figure}[!htb]
	\centering
	\caption{Diagrama de Robustez de Tipo de Problema}
	\includegraphics[width=15cm]{UML/robustez/7.png}
	\label{rbTipoProblema}
	Fonte: (AUTOR, 2015)
\end{figure}
\FloatBarrier

\subsection{Cadastro e Edi��o de Problema}

A figura \ref{rbCadastroProblema} demonstra as associa��es das classes de dom�nio na interface gr�fica da figura \ref{mkNovoProblema}.

\FloatBarrier
\begin{figure}[!htb]
	\centering
	\caption{Diagrama de Robustez de Cadastro e Edi��o de Problema}
	\includegraphics[width=15cm]{UML/robustez/8.png}
	\label{rbCadastroProblema}
	Fonte: (AUTOR, 2015)
\end{figure}
\FloatBarrier

\newpage

\subsection{Palavras-chave}

A figura \ref{rbPalavraChave} demonstra as associa��es das classes de dom�nio na interface gr�fica da figura \ref{mkPalavrasChaves}.

\FloatBarrier
\begin{figure}[!htb]
	\centering
	\caption{Diagrama de Robustez de Palavras-chave}
	\includegraphics[width=15cm]{UML/robustez/9.png}
	\label{rbPalavraChave}
	Fonte: (AUTOR, 2015)
\end{figure}
\FloatBarrier

\subsection{Dados de Testes}

A figura \ref{rbDadosTestes} demonstra as associa��es das classes de dom�nio na interface gr�fica da figura \ref{mkDadosTestes}.

\FloatBarrier
\begin{figure}[!htb]
	\centering
	\caption{Diagrama de Robustez de Dados de Testes}
	\includegraphics[width=15cm]{UML/robustez/10.png}
	\label{rbDadosTestes}
	Fonte: (AUTOR, 2015)
\end{figure}
\FloatBarrier

\newpage

\subsection{Visualiza��o de Solu��o partindo da listagem de Problemas}

A figura \ref{rbVisualizacaoSolucao1} demonstra as associa��es das classes de dom�nio na interface gr�fica da figura \ref{mkVisualizacaoSolucao1}.

\FloatBarrier
\begin{figure}[!htb]
	\centering
	\caption{Diagrama de Robustez de Visualiza��o de Solu��o partindo da listagem de Problemas}
	\includegraphics[width=15cm]{UML/robustez/11.png}
	\label{rbVisualizacaoSolucao1}
	Fonte: (AUTOR, 2015)
\end{figure}
\FloatBarrier

\subsection{Lista de Grupos}

A figura \ref{rbListaGrupos} demonstra as associa��es das classes de dom�nio na interface gr�fica da figura \ref{mkListaGrupos}.

\FloatBarrier
\begin{figure}[!htb]
	\centering
	\caption{Diagrama de Robustez de Lista de Grupos}
	\includegraphics[width=15cm, height=8.5cm]{UML/robustez/12.png}
	\label{rbListaGrupos}
	Fonte: (AUTOR, 2015)
\end{figure}
\FloatBarrier

\newpage

\subsection{Cadastro e Edi��o de Grupo}

A figura \ref{rbCadastroGrupo} demonstra as associa��es das classes de dom�nio na interface gr�fica da figura \ref{mkCadastroGrupo}.

\FloatBarrier
\begin{figure}[!htb]
	\centering
	\caption{Diagrama de Robustez de Cadastro e Edi��o de Grupo}
	\includegraphics[width=15cm]{UML/robustez/13.png}
	\label{rbCadastroGrupo}
	Fonte: (AUTOR, 2015)
\end{figure}
\FloatBarrier

\subsection{Adicionar Participantes}

A figura \ref{rbVisualizacaoSolucao2} demonstra as associa��es das classes de dom�nio na interface gr�fica da figura \ref{mkParticipantes}.

\FloatBarrier
\begin{figure}[!htb]
	\centering
	\caption{Diagrama de Robustez de Adicionar Participantes}
	\includegraphics[width=15cm]{UML/robustez/14.png}
	\label{rbVisualizacaoSolucao2}
	Fonte: (AUTOR, 2015)
\end{figure}
\FloatBarrier

\newpage

\subsection{Adicionar Lista de Problemas}

A figura \ref{rbListaProblema} demonstra as associa��es das classes de dom�nio na interface gr�fica da figura \ref{mkGrupoListaProblemas}.

\FloatBarrier
\begin{figure}[!htb]
	\centering
	\caption{Diagrama de Robustez de Adicionar Lista de Problemas}
	\includegraphics[width=15cm]{UML/robustez/15.png}
	\label{rbListaProblema}
	Fonte: (AUTOR, 2015)
\end{figure}
\FloatBarrier

\subsection{Visualiza��o de Solu��o partindo da listagem de Grupos}

A figura \ref{rbVisualizacaoSolucao3} demonstra as associa��es das classes de dom�nio na interface gr�fica da figura \ref{mkVisualizacaoSolucao2}.

\FloatBarrier
\begin{figure}[!htb]
	\centering
	\caption{Diagrama de Robustez de Visualiza��o de Solu��o partindo da listagem de Grupos}
	\includegraphics[width=15cm]{UML/robustez/16.png}
	\label{rbVisualizacaoSolucao3}
	Fonte: (AUTOR, 2015)
\end{figure}
\FloatBarrier

\newpage

\subsection{Lista de Alunos}

A figura \ref{rbListaAlunos} demonstra as associa��es das classes de dom�nio na interface gr�fica da figura \ref{mkListaAlunos}.

\FloatBarrier
\begin{figure}[!htb]
	\centering
	\caption{Diagrama de Robustez de Lista de Alunos}
	\includegraphics[width=15cm]{UML/robustez/17.png}
	\label{rbListaAlunos}
	Fonte: (AUTOR, 2015)
\end{figure}
\FloatBarrier

\newpage

\subsection{Cadastro de Alunos pelo Administrador}

A figura \ref{rbCadastroAlunoAdministrador} demonstra as associa��es das classes de dom�nio na interface gr�fica da figura \ref{mkCadastroAlunoAdmin}.

\FloatBarrier
\begin{figure}[!htb]
	\centering
	\caption{Diagrama de Robustez de Cadastro de Alunos pelo Administrador}
	\includegraphics[width=15cm]{UML/robustez/18.png}
	\label{rbCadastroAlunoAdministrador}
	Fonte: (AUTOR, 2015)
\end{figure}
\FloatBarrier

\subsection{Visualiza��o de Solu��o partindo da listagem de Alunos}

A figura \ref{rbVisualizacaoSolucao4} demonstra as associa��es das classes de dom�nio na interface gr�fica da figura \ref{mkVisualizacaoSolucao3}.

\FloatBarrier
\begin{figure}[!htb]
	\centering
	\caption{Diagrama de Robustez de Visualiza��o de Solu��o partindo da listagem de Alunos}
	\includegraphics[width=15cm]{UML/robustez/19.png}
	\label{rbVisualizacaoSolucao4}
	Fonte: (AUTOR, 2015)
\end{figure}
\FloatBarrier

\newpage

\subsection{Lista de Problemas}

A figura \ref{rbListaProblemas} demonstra as associa��es das classes de dom�nio na interface gr�fica da figura \ref{mkListaProblemasListagem}.

\FloatBarrier
\begin{figure}[!htb]
	\centering
	\caption{Diagrama de Robustez de Lista de Problemas}
	\includegraphics[width=15cm]{UML/robustez/20.png}
	\label{rbListaProblemas}
	Fonte: (AUTOR, 2015)
\end{figure}
\FloatBarrier

\newpage

\subsection{Cadastro e Edi��o de Lista de Problemas}

A figura \ref{rbCadastroListaProblemas} demonstra as associa��es das classes de dom�nio na interface gr�fica da figura \ref{mkCadastroListaProblemas}.

\FloatBarrier
\begin{figure}[!htb]
	\centering
	\caption{Diagrama de Robustez de Cadastro e Edi��o de Lista de Problemas}
	\includegraphics[width=15cm]{UML/robustez/21.png}
	\label{rbCadastroListaProblemas}
	Fonte: (AUTOR, 2015)
\end{figure}
\FloatBarrier

\subsection{Visualiza��o de Solu��o partindo da listagem de Listas de Problemas}

A figura \ref{rbVisualizacaoSolucao5} demonstra as associa��es das classes de dom�nio na interface gr�fica da figura \ref{mkVisualizacaoSolucao4}.

\FloatBarrier
\begin{figure}[!htb]
	\centering
	\caption{Diagrama de Robustez de Visualiza��o de Solu��o partindo da listagem de Listas de Problemas}
	\includegraphics[width=15cm]{UML/robustez/22.png}
	\label{rbVisualizacaoSolucao5}
	Fonte: (AUTOR, 2015)
\end{figure}
\FloatBarrier

\newpage

\subsection{Lista de Usu�rios}

A figura \ref{rbListaUsuarios} demonstra as associa��es das classes de dom�nio na interface gr�fica da figura \ref{mkListaUsuario}.

\FloatBarrier
\begin{figure}[!htb]
	\centering
	\caption{Diagrama de Robustez de Lista de Usu�rios}
	\includegraphics[width=15cm]{UML/robustez/23.png}
	\label{rbListaUsuarios}
	Fonte: (AUTOR, 2015)
\end{figure}
\FloatBarrier

\subsection{Cadastro e Edi��o de Usu�rios}

A figura \ref{rbCadastroUsuarios} demonstra as associa��es das classes de dom�nio na interface gr�fica da figura \ref{mkCadastroUsuario}.

\FloatBarrier
\begin{figure}[!htb]
	\centering
	\caption{Diagrama de Robustez de Cadastro e Edi��o de Usu�rios}
	\includegraphics[width=15cm]{UML/robustez/24.png}
	\label{rbCadastroUsuarios}
	Fonte: (AUTOR, 2015)
\end{figure}
\FloatBarrier

\newpage

\subsection{Lista de Administradores}

A figura \ref{rbListaAdministradores} demonstra as associa��es das classes de dom�nio na interface gr�fica da figura \ref{mkListaAdministradores}.

\FloatBarrier
\begin{figure}[!htb]
	\centering
	\caption{Diagrama de Robustez de Lista de Administradores}
	\includegraphics[width=15cm]{UML/robustez/25.png}
	\label{rbListaAdministradores}
	Fonte: (AUTOR, 2015)
\end{figure}
\FloatBarrier

\subsection{Cadastro e Edi��o de Administradores}

A figura \ref{rbCadastroAdministradores} demonstra as associa��es das classes de dom�nio na interface gr�fica da figura \ref{mkCadastroAdministradores}.

\FloatBarrier
\begin{figure}[!htb]
	\centering
	\caption{Diagrama de Robustez de Cadastro e Edi��o de Administradores}
	\includegraphics[width=15cm]{UML/robustez/26.png}
	\label{rbCadastroAdministradores}
	Fonte: (AUTOR, 2015)
\end{figure}
\FloatBarrier

\subsection{Lista de Professores}

A figura \ref{rbListaProfessores} demonstra as associa��es das classes de dom�nio na interface gr�fica da figura \ref{mkListaProfessores}.

\FloatBarrier
\begin{figure}[!htb]
	\centering
	\caption{Diagrama de Robustez de Lista de Professores}
	\includegraphics[width=15cm]{UML/robustez/27.png}
	\label{rbListaProfessores}
	Fonte: (AUTOR, 2015)
\end{figure}
\FloatBarrier

\subsection{Cadastro e Edi��o de Professores}

A figura \ref{rbCadastroProfessores} demonstra as associa��es das classes de dom�nio na interface gr�fica da figura \ref{mkCadastroProfessores}.

\FloatBarrier
\begin{figure}[!htb]
	\centering
	\caption{Diagrama de Robustez de Cadastro e Edi��o de Professores}
	\includegraphics[width=15cm]{UML/robustez/28.png}
	\label{rbCadastroProfessores}
	Fonte: (AUTOR, 2015)
\end{figure}
\FloatBarrier

\newpage

\subsection{Lista de Institui��es}

A figura \ref{rbListaInstituicao} demonstra as associa��es das classes de dom�nio na interface gr�fica da figura \ref{mkListaInstituicao}.

\FloatBarrier
\begin{figure}[!htb]
	\centering
	\caption{Diagrama de Robustez de Lista de Institui��es}
	\includegraphics[width=15cm]{UML/robustez/29.png}
	\label{rbListaInstituicao}
	Fonte: (AUTOR, 2015)
\end{figure}
\FloatBarrier

\subsection{Cadastro e Edi��o de Institui��es}

A figura \ref{rbCadastroInstituicao} demonstra as associa��es das classes de dom�nio na interface gr�fica da figura \ref{mkCadastroInstituicao}.

\FloatBarrier
\begin{figure}[!htb]
	\centering
	\caption{Diagrama de Robustez de Cadastro e Edi��o de Institui��es}
	\includegraphics[width=15cm]{UML/robustez/30.png}
	\label{rbCadastroInstituicao}
	Fonte: (AUTOR, 2015)
\end{figure}
\FloatBarrier

\newpage

\subsection{Lista de Permiss�es}

A figura \ref{rbListaPermissoes} demonstra as associa��es das classes de dom�nio na interface gr�fica da figura \ref{mkListaPermissao}.

\FloatBarrier
\begin{figure}[!htb]
	\centering
	\caption{Diagrama de Robustez de Lista de Permiss�es}
	\includegraphics[width=15cm]{UML/robustez/31.png}
	\label{rbListaPermissoes}
	Fonte: (AUTOR, 2015)
\end{figure}
\FloatBarrier

\subsection{Cadastro e Edi��o de Permiss�es}

A figura \ref{rbCadastroPermissoes} demonstra as associa��es das classes de dom�nio na interface gr�fica da figura \ref{mkCadastroPermissao}.

\FloatBarrier
\begin{figure}[!htb]
	\centering
	\caption{Diagrama de Robustez de Cadastro e Edi��o de Permiss�es}
	\includegraphics[width=15cm]{UML/robustez/32.png}
	\label{rbCadastroPermissoes}
	Fonte: (AUTOR, 2015)
\end{figure}
\FloatBarrier

\newpage

\subsection{Lista de Grupos de Administradores}

A figura \ref{rbListaGruposAdministradores} demonstra as associa��es das classes de dom�nio na interface gr�fica da figura \ref{mkListaGrupoAdministradores}.

\FloatBarrier
\begin{figure}[!htb]
	\centering
	\caption{Diagrama de Robustez de Lista de Grupos de Administradores}
	\includegraphics[width=15cm]{UML/robustez/33.png}
	\label{rbListaGruposAdministradores}
	Fonte: (AUTOR, 2015)
\end{figure}
\FloatBarrier

\subsection{Cadastro e Edi��o de Grupos de Administradores}

A figura \ref{CadastroGruposAdministradores} demonstra as associa��es das classes de dom�nio na interface gr�fica da figura \ref{mkCadastroGrupoAdministradores}.

\FloatBarrier
\begin{figure}[!htb]
	\centering
	\caption{Diagrama de Robustez de Cadastro e Edi��o de Grupos de Administradores}
	\includegraphics[width=15cm]{UML/robustez/34.png}
	\label{CadastroGruposAdministradores}
	Fonte: (AUTOR, 2015)
\end{figure}
\FloatBarrier

\newpage

\subsection{Lista de Pa�ses, Estados e Cidades}

A figura \ref{rbListaPaises} demonstra as associa��es das classes de dom�nio na interface gr�fica da figura \ref{mkListaPaises}.

\FloatBarrier
\begin{figure}[!htb]
	\centering
	\caption{Diagrama de Robustez de Lista de Pa�ses, Estados e Cidades}
	\includegraphics[width=15cm]{UML/robustez/35.png}
	\label{rbListaPaises}
	Fonte: (AUTOR, 2015)
\end{figure}
\FloatBarrier

\subsection{Cadastro e Edi��o de Pa�s}

A figura \ref{rbCadastriPais} demonstra as associa��es das classes de dom�nio na interface gr�fica da figura \ref{mkCadastroPais}.

\FloatBarrier
\begin{figure}[!htb]
	\centering
	\caption{Diagrama de Robustez de Cadastro e Edi��o de Pa�s}
	\includegraphics[width=15cm]{UML/robustez/36.png}
	\label{rbCadastriPais}
	Fonte: (AUTOR, 2015)
\end{figure}
\FloatBarrier

\newpage

\subsection{Cadastro e Edi��o de Estado}

A figura \ref{rbCadastroEstado} demonstra as associa��es das classes de dom�nio na interface gr�fica da figura \ref{mkCadastroEstado}.

\FloatBarrier
\begin{figure}[!htb]
	\centering
	\caption{Diagrama de Robustez de Cadastro e Edi��o de Estado}
	\includegraphics[width=15cm]{UML/robustez/37.png}
	\label{rbCadastroEstado}
	Fonte: (AUTOR, 2015)
\end{figure}
\FloatBarrier

\subsection{Cadastro e Edi��o de Cidade}

A figura \ref{rbCadastroCidade} demonstra as associa��es das classes de dom�nio na interface gr�fica da figura \ref{mkCadastroCidade}.

\FloatBarrier
\begin{figure}[!htb]
	\centering
	\caption{Diagrama de Robustez de Cadastro e Edi��o de Cidade}
	\includegraphics[width=15cm, height=8cm]{UML/robustez/38.png}
	\label{rbCadastroCidade}
	Fonte: (AUTOR, 2015)
\end{figure}
\FloatBarrier

\newpage

\section{Diagramas de Sequ�ncia}

A figura \ref{sqCadastro} demonstra o fluxo de cadastro de um novo problema. O administrador acessa a listagem de problemas, selecionando na interface para o cadastro de um  novo problema, o sistema exibe o formul�rio de cadastro e o administrador preenche todos os campos mandat�rios e clica em ``salvar" \ fazendo com que o sistema retorne para a listagem de problemas.

\FloatBarrier
\begin{figure}[!htb]
	\centering
	\caption{Diagrama de Sequ�ncia de Cadastro e Edi��o de Problemas}
	\includegraphics[width=15cm]{UML/sequencia/1.png}
	\label{sqCadastro}
	Fonte: (AUTOR, 2015)
\end{figure}
\FloatBarrier

N�o foram constru�dos outrso diagramas de sequ�ncia, porque todos seriam muito semelhantes, trocando-se somente as classes envolvidas.

\newpage

\section{Arquitetura do Software}

Para o desenvolvimento da evolu��o do portal de algoritmos foi definido utilizar a arquitetura de \textit{software} em camadas, estas representadas na figura \ref{imgArquitetura}.

\FloatBarrier
\begin{figure}[!htb]
	\centering
	\caption{Arquitetura do Portal de Algoritmos}
	\includegraphics[width=7cm]{imagens/arquitetura.png}
	\label{imgArquitetura}
	\\
	Fonte: (AUTOR, 2015)
\end{figure}
\FloatBarrier

Na camada de Apresenta��o utilizaremos \ac{HTML} e \textit{AngularJS} para a constru��o das interfaces gr�ficas, essa camada comunica-se diretamente com a camada JAX-RS, essa por sua vez respons�vel por fazer a transfer�ncia de informa��o via \ac{API} \ac{REST}.

J� a camada de Neg�cio � respons�vel pelas regras de neg�cio, essas j� definidas nos casos de uso. 

Por sua vez temos a camada de Persist�ncia, essa respons�vel por persistir e recuperar os dados, nessa camada utilizaremos \ac{DAO} para realizar essa tarefa.

A camada de banco de dados � respons�vel pelo armazenamento dos dados da aplica��o.
\chapter{Seguran�a}\label{seguranca}

O Portal de Algoritmos desatualizado apresenta algumas falhas de seguran�a, abaixo s�o citadas apenas algumas dessas falhas. Ao final � apresetando uma indica��o para aumentar a seguran�a das informa��es \ac{HTTPS}.

\section{Man in the Middle}

O ataque \textit{man-in-the middle} intercepta uma comunica��o entre dois sistemas. Por exemplo, em uma transa��o \ac{HTTP}, o destino � a conex�o \ac{TCP} entre cliente e servidor. Usando t�cnicas diferentes, o invasor divide a conex�o \ac{TCP} original em duas novas conex�es, uma entre o cliente e o invasor e a outra entre o invasor e o servidor. Quando a conex�o \ac{TCP} � interceptada, o invasor age como um \textit{proxy}, sendo capaz de ler, inserir e modificar os dados na comunica��o interceptada.\cite{OwaspMainInTheMiddle}

\section{SQL Injection}

Um ataque de inje��o SQL consiste em inser��o ou "inje��o" de uma consulta SQL por meio dos dados de entrada do cliente para o aplicativo. Uma explora��o de inje��o SQL bem-sucedida pode ler dados confidenciais do banco de dados, modificar dados do banco de dados dentre eles, inserir, atualizar e excluir. Os ataques de inje��o de SQL s�o um tipo de ataque de inje��o , no qual os comandos SQL s�o injetados na entrada do plano de dados para efetuar a execu��o de comandos SQL predefinidos.\cite{OwaspSQLInjection}

\section{HTTPS}

O \ac{HTTPS} tecnicamente falando � \ac{HTTP} sobre \ac{SSL}, ele codifica e decodifica solicita��es de p�ginas de usu�rios. � importante saber que ele protege contra ataques no site.

Para adicionar essa camada de seguran�a foi estudado o \textit{Let's Encrypt}, ele � uma autoridade certificadora livre, autorizada e aberta. Eles fornecem certificados v�lidos gratuitamente.\cite{LetsEncrypt}

\section{Sugest�o de instala��o}

No mercado existem diversas autoridades certificadoras que fornecem certificados digitais para fornecer seguran�a e criptografia de dados nos sistemas online.

Bem como existem certificados pagos, exitem tamb�m certificados gratuitos. E para esse trabalho sugerimos utilizar a \textit{Let's Encrypt}, uma autoridade certitifadora que fornece certificados gratuitos com validade de 3 meses, mas que podem seer auto-renovados diretamente no servidor da aplica��o.
\chapter{Cap�tulo 6}\label{cpCapitulo6}
\chapter{Cap�tulo 7}\label{cpCapitulo7}
\chapter{Cap�tulo 8}\label{cpCapitulo8}
\chapter{Cap�tulo 9}\label{cpCapitulo9}

\bibliography{tcc}
\bibliographystyle{abnt}

\end{document}