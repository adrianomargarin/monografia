\chapter{Considera��es Finais}\label{cpConsideracoesParciais}

O objetivo do trabalho era realizar a implementa��o de um \textit{software} novo visando a evolu��o do antigo. Esse objetivo foi entregue com alta qualidade e organiza��o de c�digo. Durante o o trabalho foram encontrado diversos problemas no \textit{software} antigo, esses problemas foram sendo resolvido de diversas formas, como as citadas abaixo.

O portal antigo utilizava tecnologias antigas, como \textit{Java Applet}, \ac{HTML} sem estrutura��o e sem valida��es, esse problema foi resolvido utilizando \textit{AngularJS 4}, um \textit{framework javascript} para implementar o consumo de \ac{API} \ac{REST}, al�m do consumo de dados � preciso exibir esses dados, ent�o foi utilizado um \textit{framework} de \ac{CSS}, esse conhecido como Bootstrap, com ele foi poss�vel criar interface responsivas, bonitas e intuitivas para o uso do \textit{software}.

O novo \textit{software} foi desenvolvido utilizando tecnologias de ponta, como \textit{Java} na �ltima vers�o est�vel, \textit{AngulaJS} na vers�o mais recente est�vel e um \textit{framework} de \ac{CSS} f�cil de utilizar.

Com a nova arquitetura do \textit{software} ficou mais f�cil e r�pido de implementar novas funcionalidades conforme as necessidades forem surgindo. A arquiteruta engloba uma \ac{API} \ac{REST} f�cil de integrar e configurar.

A interface gr�fica ficou mais amig�vel e intuitiva para o usu�rio, utilizando padr�es de telas, bot�es, listagem e cadastros em geral.

Com um banco de dados atual o \textit{software} est� mais confi�vel no armazenamento das informa��es cadastrais, com essa atualiza��o estamos garantindo integridade e seguran�a nas informa��es armazenadas nele.

Com a implementa��o do novo gerenciador do portal de algoritmos concluimos que foi construido um \textit{software} robusto, organizado, com alta qualidade de c�digo e de f�cil entendimento sobre o que ele foi proposto.

Chegamos no final desse trabalho com um \textit{software} bem escrito, estruturado e organizado com documenta��o de todas as chamadas das \ac{API}\textit{s}.
