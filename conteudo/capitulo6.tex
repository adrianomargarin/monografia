\chapter{Considera��es Finais}\label{cpConsideracoesParciais}

Problema e solu��o

% usabilidade
% tecnologia defassada

% objetivos x o que foi feito
% analise
% proposta
% desenvolvimento


O novo \textit{software} foi desenvolvido utilizando tecnologias de ponta, como \textit{Java} na �ltima vers�o est�vel, \textit{AngulaJS} na vers�o mais recente est�vel e um \textit{framework} de \ac{CSS} f�cil de utilizar.

Com a nova arquitetura do \textit{software} ficou mais f�cil e r�pido de implementar novas funcionalidades conforme as necessidades forem surgindo. A arquiteruta engloba uma \ac{API} \ac{REST} f�cil de integrar e configurar.

A interface gr�fica ficou mais amig�vel e intuitiva para o usu�rio, utilizando padr�es de telas, bot�es, listagem e cadastros em geral.

Com um banco de dados atual o \textit{software} est� mais confi�vel no armazenamento das informa��es cadastrais, com essa atualiza��o estamos garantindo integridade e seguran�a nas informa��es armazenadas nele.

Com a implementa��o do novo gerenciador do portal de algoritmos concluimos que foi construido um \textit{software} robusto, organizado, com alta qualidade de c�digo e de f�cil entendimento sobre o que ele foi proposto.
