\chapter{Seguran�a}\label{seguranca}

O Portal de Algoritmos antes da atualiza��o apresenta algumas falhas de seguran�a, abaixo s�o citadas apenas algumas como sendo graves. Al�m das falhas citadas � proposto uma solu��o utilizando \ac{HTTPS}.

\section{Man in the Middle}

O ataque \textit{man-in-the middle} intercepta uma comunica��o entre dois sistemas. Por exemplo, em uma transa��o \ac{HTTP}, o destino � a conex�o \ac{TCP} entre cliente e servidor. Usando t�cnicas diferentes, o invasor divide a conex�o \ac{TCP} original em duas novas conex�es, uma entre o cliente e o invasor e a outra entre o invasor e o servidor. Quando a conex�o \ac{TCP} � interceptada, o invasor age como um \textit{proxy}, sendo capaz de ler, inserir e modificar os dados na comunica��o interceptada.\cite{OwaspMainInTheMiddle}

\section{SQL Injection}

Um ataque de inje��o SQL consiste em inser��o ou "inje��o" de uma consulta SQL por meio dos dados de entrada do cliente para o aplicativo. Uma explora��o de inje��o SQL bem-sucedida pode ler dados confidenciais do banco de dados, modificar dados do banco de dadosm dentre eles, inserir, atualizar e excluir. Os ataques de inje��o de SQL s�o um tipo de ataque de inje��o , no qual os comandos SQL s�o injetados na entrada do plano de dados para efetuar a execu��o de comandos SQL predefinidos.\cite{OwaspSQLInjection}

\section{HTTPS}

O \ac{HTTPS} tecnicamente falando � \ac{HTTP} sobre \ac{SSL}, ele codifica e decodifica solicita��es de p�ginas de usu�rios. � importante saber que ele protege contra ataques no site.

Para adicionar essa camada de seguran�a foi estudado o \textit{Let's Encrypt}, ele � uma autoridade certificadora livre, autorizada e aberta. Eles fornecem certificados v�lidos gratuitamente.\cite{LetsEncrypt}

\textit{Let's Encrypt} possui os seguintes principios: ser gratuito, autom�tico, seguro, transparente, aberto e cooperativo.
